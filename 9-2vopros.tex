\section{Переводческие трансформации. Лексико-грамматические трансформации}

Переводческие трансформации --- это межъязыковые преобразования, осуществляемые переводчиком для преодоления несоответствия, обнаруженного в тексте оригинала по отношению к тексту перевода.

П.Т. могут производиться на лексическом и грамматическом уровнях:

Лексические - приемы, с помощью которых переводчик раскрывает значение слова ИЯ в контексте и находит соответствие в ПЯ, не совпадающее со словарным. Переводчик самостоятельно создает вариант перевода, подходящий лишь для данного конкретного случая (контекстуальная замена).

Грамматические --- межъязыковые преобразования при переводе, затрагивающие область грамматики

Лексико-грамматические трансформации --- преобразования, осуществляемые как в области грамматики, так и в области лексики. 

Причины лексико-грамматических трансформаций:

Различия в способе описания предметной ситуации. В значении слова в разных языках часто выделяются разные признаки одного и того же явления или понятия, где отражено видение мира, свойственное данному языку, вернее, носителям данного языка, что неизбежно создает трудности при переводе.

Сравним, например, glasses и очки. В английском слове выделяется материал, из которого сделан предмет, а в русском --- его функция: «вторые глаза» (очи).

Или: Hot milk with skin on it  --- Горячее молоко с пенкой.

Данное явление ассоциируется в английском языке с кожей, кожицей, покрывающей тело или плод, тогда как в русском языке в основу значения слова положен результат кипения --- пенка появляется, когда молоко кипит и пенится. Но, несмотря на выделение различных признаков, оба языка в равной степени адекватно отражают одно и то же. 

Лексико-грамматические различия ИЯ и ПЯ (Исходный Язык, Язык Перевода). Набор различий из двух областей. К примеру, с точки зрения грамматике, в языке подлинника может встречаться элемент, которому нет формально-грамматического соответствия в языке перевода (например, наличие артикля). А с точки зрения лексики можно натолкнуться на «ложных друзей переводчика», которые могут произноситься одинаково как в ИЯ, так и в ПЯ, но иметь совершенно разное значение в каждом из языков. 

Различия в фоновых знаниях носителей ИЯ и ПЯ. Третьей причиной, вызывающей необходимость в трансформациях, является возможное недопонимание между автором и переводчиком, точнее незнание вторым каких либо норм, традиций, которые нормальны для первого

Stuart cooked piperade --- Стюарт приготовил piperade 

Здесь используется название традиционного баскского блюда, которое не имеет аналога в русской кухне и, следовательно, не переводится на русский язык. Баски --- это народ, говорящий на изолированном языке, составляющий коренное население северо-западных областей Испании и юго-западных районов Франции, и, следовательно, это кушанье широко распространено только среди жителей данных регионов Испании и Франции. Переводчик должен быть хорошо осведомлен о подобных особенностях.

Основные типы лексико-грамматических трансформаций

Добавления --- восстановление в ПЯ формально невыраженных (выраженных имплицитно) элементов ИЯ. Лексические трансформации нередко требуют внесения дополнительных слов. Введение дополнительных слов часто обусловливается тем, что более сжатые английские предложения требуют в русском языке более полного выражения мысли. 

I saw the new Leonardo DiCaprio movie --- Я посмотрел новый фильм, в котором снимался Леонардо Ди Каприо 

Where are you from? --- Seattle, Washington --- Откуда вы? --- Из Сиэтла, штат Вашингтон 

Опущения --- преобразование, в результате которого опускаются некоторые компоненты исходного высказывания, которые не несут важной смысловой нагрузки, а их значение зачастую восстанавливается в переводе. Опущение является методом прямо противоположным приему добавления.

Пожалуй, самым традиционным примером избыточности является употребление так называемых «парных синонимов», часто проявляющееся во всех стилях письменной речи английского языка. Однако в русском языке такое проявление не встречается, и при его переводе один из синонимов не повторяется, и два слова заменяются одним. Так и используется прием опущения. Например: brave and fearless --- храбрый; just and equitable treatment --- справедливое отношение

In all towns and cities… ---  Во всех городах

Безусловно, этот прием находит применение не только при переводе «парных синонимов», имеются и другие избыточные элементы. К примеру:

He brushed his teeth --- Он почистил зубы

Антонимический перевод --- лексико-грамматическая трансформация, основанная на противопоставлении понятий: отрицание понятия в высказывании ИЯ приравнивается к утверждению противоположного понятия в высказывании ПЯ (и наоборот)

He didn’t say anything --- Он промолчал

Hang on, please --- Не кладите трубку, пожалуйста 

Take your time --- Не торопись

В англо-русских переводах эта трансформация применяется особенно часто, когда в оригинале отрицательная форма употреблена со словом, имеющим отрицательный префикс:

She is not unworthy of your attention. ---  Она вполне заслуживает вашего внимания.

Сюда относится и употребление отрицательной формы с отрицательными союзами until и unless:

The United States did not enter the war until April 1917. --- Соединенные Штаты вступили в войну только в апреле 1917 г.

Следует учитывать, что отрицание может выражаться и другими средствами, например, при помощи союза without:

Не never came home without bringing something for the kids. ---  Приходя домой, он всегда приносил что-нибудь детям.

Компенсация --- применяется в тех случаях, когда определенные элементы текста ИЯ не имеют эквивалентов в ПЯ и не могут быть переданы его средствами. Чтобы восполнить ("компенсировать") семантическую потерю, вызванную тем, что та или иная единица ИЯ осталась непереведенной, переводчик передает ту же самую информацию иными средствами (необязательно в том же самом месте текста, что и в подлиннике)

I’ve brought a Christmas present for Dad. --- Это папе новогодний подарок.

Некоторые особенности английского просторечия нельзя передать на русский язык никакими иными средствами, кроме компенсации, например, добавление или опущение гласных или согласных звуков, отсутствие согласования между подлежащим и сказуемым (I was, you was и пр.) или какое-либо иное нарушение грамматических правил. Иногда такая компенсация достигается относительно простым способом. В пьесе Б. Шоу «Пигмалион» Элиза говорит: I’m nothing to you $\rightarrow $ not so much as them slippers. Хиггинс поправляет ее: those slippers. Разницу между them и those трудно воспроизвести в переводе. Но эту «утрату» легко компенсировать, обыграв неправильную форму родительного падежа туфли. В переводе Элиза скажет: Я для вас ничто, хуже вот этих туфлей, а Хиггинс поправит ее: туфель. В других случаях для решения задачи придется использовать единицы ПЯ, не имеющие соответствий в оригинале:

You could tell he was very ashamed of his parents and all, because they said ‘he don’t’ and ‘she don’t’ and stuff like that. --- Было видно, что он стесняется своих родителей, потому что они говорили «хочут»  и «хочете» и все в таком роде.
