\section{Переводческие трансформации. Лексические трансформации}

Переводческие трансформации --- межъязыковые преобразования, совершаемые переводчиком для преодоления несоответствия теста оригинала и текста перевода с тем, чтобы текст перевода с максимально возможной полнотой передавал всю информацию, заключенную в исходном тексте, при строгом соблюдении норм языка перевода.

Лексические трансформации --- замена отдельных лексических единиц (слов и устойчивых словосочетаний) ИЯ лексическими единицами ПЯ, которые не являются их словарными эквивалентами. 

1. Конкретизация --- замена слова/словосочетания исходного языка с более широким значением на слово/словосочетание языка перевода с более узким значением. 

--- thing имеет очень абстрактное значение и переводится путем конкретизации: вещь, предмет, дело, факт, случай, обстоятельство, произведение, существо и пр. 

--- come и go: они, в отличие от русских глаголов движения, не указывают на способ передвижения, поэтому come при переводе конкретизируется как приходить, прибывать, приезжать, подходить, подбегать, приплывать, прилетать и пр., a go --- как идти, ходить, ехать, отправляться, сходить, проходить, плыть, лететь и пр.

--- say и tell: могут переводиться не только как говорить и (рас)сказать, но и как (про)молвить, повторить, заметить, отметить, утверждать, сообщать, высказываться, спросить, возразить, приказать, велеть и пр.

'So what?' I said. (J. Salinger, The Catcher in the Rye, 6). --- Ну так что же? спрашиваю я.

'Hello', I said when somebody answered the goddam phone, (ib., 20) --- Алло! --- крикнул я, когда кто-то подошел к этому треклятому телефону.

She had said that she was in bed and ill. (W. Thackeray, Vanity Fair, XIX) --- Бекки писала, что она больна и лежит в постели, (пер. М. Дьяконова)

Обычно конкретизации требует глагол be, например: 

Не is at school --- Он учится в школе;

Не is in the Army --- Он служит в армии;

Не was at the ceremony --- Он присутствовал на церемонии;

The concert was on Sunday--- Концерт состоялся в воскресенье.

В романе Ч. Диккенса "Давид Копперфильд" есть такой эпизод. Женщина сидит в полутемной комнате, глубоко задумавшись. Внезапно в комнату с шумом врывается эксцентричная тетушка, испугав женщину. Эту ситуацию описывает мальчик: "My mother had left her chair in agitation and gone behind it in the corner." Неприемлемость перевода: "Взволнованная матушка оставила свое кресло и пошла за него в угол» --- очевидна. Обеспечить эквивалентность перевода можно путем конкретизации глаголов "leave" и "go": «Взволнованная матушка вскочила со своего кресла и забилась в угол позади него".

2. Генерализация --- расширение объема понятия: замена единицы ИЯ, имеющей более узкое значение, единицей ПЯ с более широким значением.

Примеры:

Then this girl gets killed, because she's always speeding. А потом девушка гибнет, потому что она вечно нарушает правила.

"Who won the game? "I said. "It's only the half". --- А кто выиграл? --- спрашиваю. --- Еще не кончилось.

Различия между русскими "теща и свекровь" или "шурин и деверь" обобщаются в английских переводах: "mother-in-low" и "brother-in-low". Английское предложение: "I saw a man 6 feet 2 inches tall" может быть переведено: "Я увидел высокого парня", т.к. в художественных произведениях в русском языке не принято указывать точный рост, вес персонажей. Обобщенный перевод здесь дается с учетом стилистических особенностей.

3. Логическое развитие (модуляция) --- замена слова или словосочетания ИЯ единицей ПЯ, значение которой логически выводится из значения исходной единицы.

"Manson climbed into the gig behind a tall horse". Нельзя сказать: "Он сел в телегу позади лошади (как будто лошадь тоже сидела в телеге)". Удачным переводом будет: "Мэнсон влез в коляску, запряженную крупной лошадью".

I don't blame them. Я их понимаю (Замена основана на причинно-следственных отношениях: я их не виню потому, что я их понимаю).

He's dead now.  Он умер. (Он умер, стало быть, он сейчас мертв.)

...If a client went to him with some trouble that was not quite nice, Mr Skinner would look grave. ...Если клиент излагал ему обстоятельства, которые могли показаться неблаговидными, мистер Скиннер озабоченно сдвигал брови, (пер. Е. Калашниковой) (Сдвигал брови, поэтому выглядел хмурым.)

4. Целостное преобразование --- трансформации повергается не отдельное слово, а смысловой комплекс (словосочетание/предложение); ни один из элементов преобразуемого комплекса по отдельности по смыслу не связан с новой формой выражения ПЯ. Специфика живой разговорной речи чаще всего требует целостного преобразования при переводе.

Help yourself --- Угощайтесь, My pleasure --- Пожалуйста, Welcome! --- добро пожаловать, Never mind --- ничего, не беспокойтесь, Don’t mention it --- не обращайте внимания, Forget it --- не стоит благодарности, не стоит говорить об этом, Here you are --- вот, пожалуйста, Well done --- браво, молодец.