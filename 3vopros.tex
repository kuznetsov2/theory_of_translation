\section{Эквивалентность перевода}

Эквивалентность перевода --- это общность содержания (смысловая близость), равноценность текстов оригинала и перевода.

И.С. Алексеева: современное понимание эквивалентности текстов состоит в достижении максимально возможного подобия двух текстов, на которое только способен самый квалифицированный переводчик, а не их тождества/перевод не копия оригинала в буквальном смысле слова.

Большинство исследователей полагают, что абсолютная эквивалентность (тождественность) исходного и переводного текстов невозможна вследствие семантических, структурных и прагматических различий между исходным текстом и текстом перевода, и признают относительность реально достижимой эквивалентности перевода.

По Я.И. Рецкеру понятие эквивалентности распространялось лишь на отношения между микроединицами текста, но не на межтекстовые отношения. При этом эквивалент понимался как постоянное равнозначное соответствие, как правило, независящее от контекста.

Такое узкое понимание эквивалента объясняется местом этой категории в используемой в теории закономерных соответствий системе понятий. Ведь родовым понятием в этой системе является соответствие, а видовыми --- эквивалент и вариантное соответствие, устанавливаемое между словами в том случае, когда в языке перевода существует несколько слов для передачи одного итого же значения исходного слова.

Вариантные соответствия: когда слово многозначно, то ни одно из них нельзя считать эквивалентным (Рецкер приводит слово pin для примера, у которого при переводе более 10 только технических значений: булавка, шпилька, штифт, болт, ось и т.д.).  

Частичные эквиваленты: Слово shadow имеет частичный эквивалент тень, соответствующий его основному значению (побочным значениям соответствуют рус. полумрак и призрак).\\
(Моё примечание: Большой разницы между частичными эквивалентами и вариантными соответствиями не обнаружено).

Полные эквиваленты: географические названия, собственные имена, термины.

В.Н. Комиссаров в своей книге «Теория перевода (лингвистические аспекты)» сформулировал теорию уровней эквивалентности, согласно которой в процессе перевода устанавливаются отношения эквивалентности между соответствующими уровнями оригинала и перевода. В.Н. Комиссаров выделил в плане содержания оригинала и перевода пять уровней эквивалентности: 

1. Формальный уровень/уровень знаков (слов) --- подстановка одних знаков другими.

В переводе сохраняются все основные части содержания оригинала: цель коммуникации, указание на ту же предметную ситуацию, способ ее описания, параллельная синтаксическая структура, лексическое подобие.

Высокая степень параллелизма в структурной организации текста.

Максимальная соотнесенность лексического состава: можно указать соответствия всем знаменательным словам оригинала.

Сохранение в переводе всех основных частей содержания.

I saw him at the theatre. --- Я видел его в театре.

The house was sold for eighty thousand dollars. --- Дом был продан за восемьдесят тысяч долларов.

2. Уровень описания ситуации --- при сохранении содержания перевода, добавляется инвариантность синтаксических структур оригинала и перевода (преобразование на уровне формы).

В переводе сохраняются цель коммуникации, указание на ту же предметную ситуацию, способ ее описания.

Значительный (хотя и неполный) параллелизм лексического состава: можно указать соответствия большинству слов оригинала или слова с близким содержанием к оригиналу.

Использование в переводе синтаксических структур, аналогичных структурам оригинала или связанных с ними отношениям синтаксического варьирования, что обеспечивает максимально возможную передачу в переводе значения синтаксических структур оригинала.

I told him what I thought of him. --- Я сказал ему свое мнение о нем. 

He was never tired of old songs. --- Старые песни ему никогда не надоедали. 

3. Уровень сообщения --- в переводе сохраняются общие понятия, с помощью которых описывается ситуация. 

В переводе сохраняются цель коммуникации, указание на ту же предметную ситуацию, способ ее описания.

Сопоставление оригинала и перевода обнаруживает отсутствие параллелизма их лексического состава и синтаксической структуры, а также сохранение в переводе общих понятий, с помощью которых осуществляется описание ситуации в оригинале. 

Scrubbing makes me bad-tempered. --- От мытья полов у меня настроение портится. 

4. Уровень высказывания 

В переводе сохраняются цель коммуникации, указание на ту же предметную ситуацию, хотя способ ее описания изменяется. 

Здесь сохранение указания на одинаковую ситуацию сопровождается в переводе структурно-семантическими расхождениями с оригиналом. 

Несопоставимость лексического состава и синтаксической организации.

Невозможность связать лексику и структуру оригинала и перевода отношениями семантического перефразирования или синтаксическо трансформации.

Сохраняется доминантная функция высказывание --- сохранение в переводе цели коммуникации.

Сохранение в переводе указания на ту же самую ситуацию.

He answered the telephone. --- Он снял трубку. 

Keep of the grass. --- По газонам не ходить.

5. Уровень цели коммуникации --- на этом уровне перевода сохраняется только та часть содержания оригинала, которая составляет цель коммуникации. 

Цель коммуникации представляет собой «переносный» смысл, присутствующий в скрытом виде и выводимый из всего высказывания как смыслового целого.

Несопоставимость лексического состава и синтаксической организации.

Невозможность связать лексику и структуру оригинала и перевода отношениями семантического перефразирования или синтаксической трансформации.

Отсутствие реальных или прямых логических связей между сообщениями в оригинале и переводе, которые позволили бы утверждать, что в обоих случаях «сообщается об одном и том же».

Наименьшая общность содержания оригинала и перевода по сравнению со всеми иными переводами, признаваемыми эквивалентными.

Maybe there is some chemistry between us that doesn`t mix. --- Бывает, что люди не сходятся характерами. 
\\
(Здесь коммуникативный эффект достигается за счет своеобразного художественного изображения человеческих отношений, уподобляемых взаимодействию химических элементов). 

That’s pretty thing to say! --- Как не стыдно!

Согласно теории В.Н. Комиссарова, эквивалентность перевода заключается в максимальной идентичности всех уровней содержания текстов оригинала и перевода. Оригинал и перевод могут быть эквивалентны друг другу на всех пяти уровнях или только на некоторых из них.

Трактовка уровней эквивалентности, предложенная А.Д. Швейцером, который считает, что эквивалентность перевода нужно оценивать на уровне всего текста.  

Главное --- какой-то один определенный аспект содержания, и поиск соответствия направлен на отражение, передачу доминантных элементов содержания.

\begin{table}[!h]
	\small
	\centering
	\caption*{По типологии Р.Якобсона:}
	\setlength{\extrarowheight}{2mm}
	\begin{tabular}{lll}
		 \toprule
		Функция             & \parbox{4cm}{Элемент коммуникации (ключевой)} & Примеры/цель                                                                                                            \\ \midrule
		Эмотивная           & Адресант                        & \parbox{8cm}{Выражает,эмоции, мысли,адресанта. Предложения будут начинаться с I think, I suppose..}                                   \\ %\midrule
		Конативная          & Адресат                         & \parbox{8cm}{Влияние, побуждение  сообщения на адресата (приказы,пропаганда, реклама)}                                                \\ %\midrule
		Референтивная       & Контекст или предмет            & \parbox{8cm}{Простые предложения передающие объективную реальность (деловая, научная коммуникация)}                                   \\ %\midrule
		Поэтическая         & Сообщение                       & \parbox{8cm}{Отношение сообщения к себе. Главным становится само сообщение, его эстетическая форма. Примеры: худ. литература, поэзия} \\ %\midrule
		Фатическая          & Контакт                         & \parbox{8cm}{Поддержка коммуникативной связи между адресатом и адресантом (пример, «Алло!»)}    \\ %\midrule
		Метакоммуникативная & Код                             & Правила языка, языковедение                    \\ \bottomrule                                                                        
	\end{tabular}
\end{table}
{\small (Моё примечание: т.е. по Швейцеру главное в эквивалентом переводе --- донести одну из функций содержания исходного текста. Если это реклама, то главным будет донести до получателя сообщения побуждающую функцию этой рекламы).}

Его теория учитывает 2 взаимосвязные признака: 1. характер трансформации, которой подвергается исходной высказывание при переводе; 2. характер сохраняемого инварианта (часть содержания оригинала, сохранение которой необходимо и достаточно для достижения эквивалентности перевода).

Выделяются синтаксический уровень: меняется субтитуция, сохраняется синтаксис (The sun disapperad behind a cloud. --- Солнце скрылось за тучей.);

--- семантический уровень эквивалентности: 

--- референциальный (лексико-грамматические трансформации) подуровень

В переводе сохраняется набор сем оригинала при расхождении формально-структурных средств выражения. An electrician has been sent for. --- Послали за электриком; 

--- компонентный (трансформации на грамматическом уровне) подуровень: сохраняется ситуация.

Описание ситуации в текстах оригинала и перевода осуществляется при помощи нетождественных семантических комплексов. He visits me practically every week-end. --- Он ездит ко мне почти каждую неделю.

При таком понимании единицы каждого вышестоящего уровня эквивалентности включают в себя единицы уровня нижеследующего, но обратной зависимости не существует. Прагматический (коммуникативный) уровень является определяющей частью   переводческого соответствия  как такового: он наслаивается на другие уровни эквивалентности и управляет ими.

Прагматический уровень:

Передача коммуникативного эффекта высказывания, для которого в сопоставляемых языках существуют свои традиционно закрепленные формы (пословицы, поговорки, устойчивые формы описания ситуаций, речевые клише, каламбуры).

Перевод на данном уровне характеризуется полной несопоставимостью лексического состава, невозможностью связать структуру оригинала и перевода отношениями семантических и синтаксических трансформаций. (Many happy returns of the day! --- С Днём Рождения!)

А.Д. Швейцер полагает, что высшую позицию в их иерархии занимает прагматический уровень. Он охватывает самые существенные для понимания текста коммуникативные факторы: коммуникативную интенцию (цель коммуникации), функциональные параметры текста, установку на адресата и коммуникативный эффект. 