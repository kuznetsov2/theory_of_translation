\section{Теория закономерных соответствий. Типы лексических соответствий}

Теория закономерных соответствий, предложенная в 1950 г. Я.И. Рецкером, является одной из первых теорий перевода, созданных в результате изучения текстов в переводе (текста оригинала и текста перевода). Рецкер различает три категории закономерных соответствий. Так, для перевода отдельных слов и словосочетаний необходимо найти:

1) существующие в другом языке однозначные эквиваленты --- постоянные соответствия, например, для имен собственных, географических названий, терминов, числительных, местоимений, наименований организаций, партий;

2) вариантные контекстуальные соответствия - аналоги, полученные с помощью выбора одного из синонимов, пригодного для данного контекста, для данного употребления слова или словосочетания в речи;

3) адекватные замены --- соответствия, выбранные исходя из целого, т.е. все виды переводческих трансформаций.

Под эквивалентами понимаются постоянные, не зависящие от контекста соответствия единиц ИЯ единицам ПЯ. Прежде всего это --- однозначные термины. Например, английскому the United Nations соответствует русское Организация Объединенных Наций. Такого рода соответствия независимы от контекста, они всегда постоянны, хотя их сравнительно немного.

Вторая группа --- аналоги, или вариантные соответствия, --- группа гораздо более многочисленная. Среди синонимических единиц ПЯ, соответствующих данной единице ИЯ, подыскивается вариант передачи смысла, наиболее подходящего для данного контекста. Другими словами, перед нами уже не однозначные соответствия, а наборы соответствующих друг другу единиц ИЯ и ПЯ. Выбор каждой конкретной пары соответствий определяется контекстом. Так, английское слово fair можно переводить на русский язык как честный и как справедливый. Словосочетание fair share мы переведем, используя слово справедливый: справедливая доля. А словосочетание fair deal --- с помощью слова честный: честная сделка. Именно контекст словосочетания, в котором употреблено английское слово, повлиял на наш выбор русского соответствия.

В группу трансформаций включены все соответствия, которые переводчик «выстраивает» сам, не пользуясь готовым арсеналом средств. Под трансформациями понимаются межъязыковые преобразования, требующие перестройки на лексическом, грамматическом или текстовом уровне. В процессе перевода встречаются трансформации 4 элементарных типов:

Перестановка --- это изменение в переводе расположения (порядка следования) языковых элементов, соответствующих языковым элементам подлинника. Наиболее частотны перестановки членов предложения --- изменение порядка слов, которое часто связано с объективными различиями в закономерностях порядка слов в русском и английском языках: I'll come late today. --- Сегодня я приду поздно.

Замена --- это наиболее распространенный вид переводческих трансформаций. Замены часто касаются формы слова (beans --- фасоль), частей речи, членов предложения, синтаксических замен в сложном предложении. На лексическом уровне также может происходить преобразование лексемы, а именно: частичное изменение или перераспределение семного состава исходной лексемы, конкретизация, генерализация.

Добавления --- расширение текста подлинника, связанное с необходимостью полноты передачи его содержания, а также с разичями в грамматическом строе. Добавления могут происходить как на лексическом, так и на грамматическом (артикль, вспомогательный глагол) уровнях.

Опущения могут охватывать избыточные компоненты традиционного словоупотребления или быть связанными с необходимостью компрессии в устном последовательном или синхронном переводе.

Однако, разграничение элементарных типов условно, так как трансформации редко встречаются в чистом виде и представляют собой, как правило, комплексные преобразования. 