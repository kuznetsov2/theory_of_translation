\section{Прагматические аспекты перевода}

Термин «Прагматика» (30 гг. XX века) --- раздел семиотики, который изучает отношения между знаками и пользователями этими знаками.

Деятельность переводчика всегда направлена на уяснение намерений автора оригинала, выраженных в тексте, и на обеспечение того воздействия на читателя (слушателя), на которое рассчитывал автор оригинала. 

Прагматика перевода --- влияние на ход и результат переводческого процесса необходимости воспроизвести прагматический потенциал оригинала и стремление обеспечить желаемое воздействие на реципиента перевода (В.Н. Комиссаров).

Основная функция любого текста заключается в создании определенного коммуникативного эффекта, то есть в оказании определенного прагматического воздействия на получателя текста. Например, художественный текст --- эстетическая функция.

При переводе большое значение имеет определение специфики, коммуникативной направленности исходного текста.

Альберт Нойберт выделяет следующие типы исходных текстов:
\\
1) Текст ИЯ не предназначен специально для аудитории ИЯ (интересен аудитории ИЯ и ПЯ) --- например научная, техническая литература.
\\
2) Текст ИЯ содержит информацию, отвечающую потребностям аудитории ИЯ: тексты законов, местная пресса, объявления.
\\
3) Художественная литература. Тексты оказываются вне временных рамок, т.к. выражают общечеловеческие ценности.
\\
4) Текст, созданный на ИЯ и предназначенный прежде всего для перевода на ПЯ. Например, информационно-пропагандистские материалы, адресованный иностранной аудитории. Также материалы изначально пишутся с учетом предполагаемой реакции.

Переводчик учитывает тип текста при переводе.

Коммуникативная направленность текста как правило соответствует типу высказывания. 

Основными требованиями, предъявляемыми к результату перевода, являются:
\\
--- текст перевода и оригинала должны выполнять одну и ту же доминантную функцию;
\\
--- коммуникативный эффект, производимый текстом перевода на своего получателя, должен быть примерно таким же, как и коммуникативный эффект, производимый текстом оригинала.

Поскольку каждая группа коммуникантов принадлежит к иноязычной, т.е. своей собственной культуре, следовательно, обладает собственным менталитетом, национальной психологией и мировосприятием. Основной прагматической установкой, по мнению Швейцера, является учет расхождений в восприятии одного и того же текста со стороны носителей разных культур, участников различных коммуникативных ситуаций. Здесь сказываются различия в исходных знаниях, представлениях, поведенческих нормах.

В процессе межъязыковой коммуникации переводчик (который не только билингвистичен, но и бикультурен) имеет возможность принять во внимание особенности потенциальных получателей текста перевода при создании перевода.

Задача обеспечения адекватного восприятия текста получателем перевода решается наиболее успешно в том случае, когда переводчику удалось достоверно определить, какая именно группа читателей (слушателей) будет выступать в качестве получателя текста перевода. 

Юджин Найда, говоря об адекватности перевода, отмечает, что когда возникает вопрос, какой их двух переводов одного текста лучше, ответ на этот вопрос следует искать в ответе на другой вопрос: «Лучше Для кого?»

Конечно, есть тексты, которые, пусть не сразу, обретают в качестве своего адресата все человечество, например, художественная литература. Здесь переводчику приходится ориентироваться на так называемого «усредненного получателя текста».

Комиссаров: «Наилучших результатов добивались переводчики, близкие по взглядам и творческой манере к автору переводимого текста». Однако, он же отмечает, что: «в современных условиях от переводчика требуется умение квалифицированно переводить тексты самых различных авторов и направлений. Ему необходимо умение перевоплощаться».

Прагматическая адаптация текста --- приведение текста в такую форму, которая максимально облегчает его восприятие и способствует оказанию соответствующего коммуникативного эффекта.

Жанровая принадлежность текста в значительной степени влияет на степень его прагматической адаптации. Например, научно-технический стиль отличается меньшим числом расхождений в разных языках, научные и технические тексты нуждаются в меньшей адаптации.

К способам прагматической адаптации текста относят:
\\
1) Эксплицирование подразумеваемой в оригинале информации путем соответствующих дополнений и пояснений.

Дополнения и пояснения используются при переводе на русский язык:
\\
--- при переводе географических названий

Например, Alberta --- канадская провинция Альберта

--- при переводе названий печатных органов, учреждений, фирм.

Например, Newsweek --- журнал «Ньюсуик»

--- при необходимости обеспечить понимание названий реалий, связанных с особенностями быта и жизни иноязычного коллектива.

Например (Комиссаров): «For dessert we got Brown Betty which nobody ate (Salinger)» --- «На сладкое подали «рыжую Бэтти» , пудинг с патокой, только его никто не ел».

--- при упоминании о структурах государственной власти.

Например (с русского на английский): при упоминании Государственной Думы в английском тексте уместно сделать пояснение: State Duma, the lower house of the Russian Parliament.

С этой же целью могут использоваться постраничные сноски, либо примечания к целому тексту перевода.

Использование примечаний уместно и тогда, когда речь идет не просто о реалиях, неизвестных читателю ПЯ, но упоминаются факты, события, явления, отражающие особенности исходной культуры.

Например (Найда пишет): многим племенам западной Африки может показаться предосудительным поведение учеников Христа, которые на его пути в Иерусалим «резали ветви с деревьев и постилали их по дороге», т.к. у них существует обычай --- дорога, по которой идет уважаемый человек, должна быть очищена от всякого мусора, и тот, кто бросит ветку, наносит тяжкое оскорбление».

2) Опущение информации, неизвестной получателю перевода и не являющейся коммуникативно релевантной:

… everything smelled like Vicks’ Nose Drops. --- … все пахло каплями от насморка.

Фирменное название Vicks' ничего не говорит русскому читателю, не несет важной информации, поэтому в переводе его можно и даже лучше опустить.

3) Использование приема генерализации, т.е. замены слова с конкретным значением словом с более общим значением, но более понятном для получателя перевода значением.

He appeared to be a young man of 6 feet and 2 inches. --- Он оказался молодым человеком выше среднего роста. (русскому читателю английская система мер малоизвестна, а следовательно, буквальный перевод хотя технически вполне возможен, не обеспечил бы передачу информации адекватно.)

Часто генерализация используется при замене имени собственного именем нарицательным, дающим родовое название для данного предмета. Например, «У него есть Роллс-Ройс» или «Он ездит на Роллс-Ройсе» (He owns a Rolls-Royce). Можно перевести как «Он ездит на шикарном, дорогущем автомобиле».

4) Использование приема конкретизации.

The British people are still profoundly divided on the issue of joining Europe. - В английском народе до сих пор существуют глубокие разногласия о том, стоило ли Англии вступать в «общий рынок». (русский читатель может не знать, в каком значении использовано слово Europe, что может означать фраза «присоединиться к Европе»).

5) Сглаживание общекультурных промахов и конфликтных ситуаций.

Все эти способы прагматической адаптации не предполагают значительного изменения содержания высказывания в переводе.

В некоторых случаях подобного рода адаптация текста недостаточна для обеспечения полной понятности текста для получателей --- представителей иной культуры. Условия жизни могут быть настолько различны, что одно и то же по содержанию высказывание вызовет абсолютно разный, если не противоположный коммуникативный эффект.

Например, сообщение о «потеплении международной обстановки» может быть понято в Индии как «обострение обстановки», поскольку для читателей этой жаркой страны приятно не потепление, а охлаждение (напрмер, welcoming cooling).

Или например выражение «белый как снег» можно передать как «белый, словно оперение цапли», если носителям ПЯ неизвестно, что такое снег.

Т.о., изменения могут быть значительны, хотя не изменяется смысл в целом.

Традиционно считается, что переводчик как языковой посредник должен быть прагматически нейтрален: однако вряд ли это полностью верно, иногда даже в корне не верно.

Переводчик может выполнять функции, никоим образом не связанные с особенностями данного акта коммуникации. Обычно предполагается, что цели автора оригинала и переводчика совпадают, или по крайней мере не разделяются, но это не совсем так.
Комиссаров: «переводчик может преследовать дополнительные цели, может стремиться использовать результат переводческого процесса в каких-либо целях. Говорят, что в
 этих случаях переводчик выполняет «прагматическую сверхзадачу».

Например, устный переводчик часто влияет на развитие коммуникации, например, направляет ход дискуссии, определяет последовательность обсуждения вопросов, напоминает о том, какой вопрос остался без ответа, разъясняет одной из сторон особенности поведения другой стороны, обусловленными особенностями культуры.

Переводчик может ставить перед собой цели пропагандистского, просветительского и т.п. характера, он может стремиться в чем-то убедить рецептора перевода, навязать свое отношение к автору оригинала или к описываемым событиям, на него могут оказывать влияние соображения политического, религиозного или личного порядка.

Т.о., сама переводческая практика доказывает, что тезис о прагматической нейтральности переводчика --- это миф, созданный в результате абстрагирования от реальной действительности.

Коммуникативный эффект, который должен быть воспроизведен в переводе, может определяться доминирующей функцией оригинала.

При этом переводчику приходится прибегать иногда к значительным изменениям. Например, при передаче на иной язык текста рекламы, который должен обеспечить сбыт данного товара, нередко приводит к составлению на ПЯ нового параллельного текста (co-writing), учитывая специфические вкусы будущих покупателей.

Проблемы вызывает передача диалектных особенностей речи и передача в переводе имитации речи иностранца, содержащейся в оригинале, особенно если это касается искажения форм. Часто переводчик использует для передачи искаженных форм какого-то общепринятого в ПЯ способа передачи «Моя твоя не понимает», «Я буду уходить», не согласованные прилагательные и существительные в роде и падеже.