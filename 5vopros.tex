\section{Норма перевода. Типы переводческих ошибок}

Норма перевода --- совокупность требований, предъявляемых к качеству перевода. Качество перевода определяется степенью его соответствия переводческой норме и характером невольных или сознательных отклонений от этой нормы. 

Норма перевода складывается в результате взаимодействия пяти различных видов нормативных требований. 

Норма эквивалентности перевода. Перед оригиналом и переводом стоит одна и та же задача, а именно определенным образом повлиять, или воздействовать, на читателя. Различие заключается лишь в том, на какого читателя оказывается влияние, или воздействие. Оригинал нацелен на читателя, владеющего языком, на котором написан оригинал. Перевод рассчитан на человека, который, не владея языком оригинала, нуждается в посреднике, с помощью которого он и знакомится с оригиналом. Таким образом, норма эквивалентности перевода означает необходимость общности содержания оригинала и перевода (при соблюдении других нормативных требований, которые вкупе обеспечивают адекватность перевода).

Жанрово-стилистическая норма перевода ---  требование соответствия перевода доминантной функции и стилистическим особенностям типа текста, к которому принадлежит перевод. (Нельзя по одинаковым критериям переводить высокохудожественную литературу и юридический документ.) Другими словами, в процессе перевода переводчик создает текст того же типа, что и оригинал.

Норма переводческой речи. Текст перевода --- это речевое произведение на Языке Перевода, и для любого переводного текста обязательны правила нормы и узуса [Узус --- общепринятое носителями данного языка употребление слов, устойчивых оборотов, форм, конструкций] ПЯ. Однако эти правила неодинаковы для всех случаев функционирования языка. Они могут варьироваться в зависимости от разновидности общелитературного языка. К примеру, различают язык разговорной речи (неформального общения) и язык художественной литературы. При этом, надо учитывать особенности ИЯ. Таким образом, норму переводческой речи можно определить как требование соблюдать правила нормы и узуса ИЯ с учетом узуальных особенностей переводных текстов на этом языке.

Прагматическая норма перевода --- требование обеспечения прагматической ценности перевода. Т.е. вызвать такую же эмоциональную/интеллектуальную реакцию читателя перевода, какую бы вызвал оригинал. Это не «норма» в полном смысле этого слова, потому что для ее выполнения переводчик может отказаться от максимальной эквивалентности, перевести оригинал частично, изменить жанровую принадлежность текста.   

Конвенциональная норма перевода --- требование максимальной близости перевода к оригиналу (его способность полноценно заменять оригинал, как в целом, так и в деталях, выполняя задачи, ради которых перевод был осуществлен на данном историческом этапе).  

Эти виды требований расположены иерархически. 
\\
1. прагматическая норма перевода
\\
2. норма переводческой речи
\\
3. жанрово-стилистическая норма перевода
\\
4. конвенциональная норма перевода
\\
5. норма эквивалентности перевода

Прежде всего, от переводчика ожидают выполнения прагматической сверхзадачи. Особые задачи  явление нечастое, обычно требования прагматической нормы удовлетворяются путем обеспечения достаточно высокого уровня эквивалентности перевода. Норма переводческой речи определяется во многом жанрово-стилистической принадлежностью оригинала и предполагает владение этим типом речи. Жанрово-стилистическая норма является заданной для большинства текстов одного типа, для специалистов по одному типу не требуется заново анализировать стилистические особенности оригинала. Конвенциональная норма долго остается неизменной и определяет общий подход к переводу. И, наконец, требование эквивалентности должно выполняться при условии соблюдения всех остальных. 

\subsection*{Типы переводческих ошибок}

Смысловые/фактические ошибки  грубое искажение смысла

Пример: I want to go, I lied. I'd always been a bad liar, but I'd been saying this lie so frequently lately that it sounded almost convincing now. 

Ошибочный вариант: --- Хочу уехать, --- твердо сказала я. Врать я всегда умела, а в последнее время так часто повторяла эти слова, что почти поверила в них сама.

Верный перевод: --- Я хочу поехать, --- соврала я. Лгунья из меня всегда была никудышная, но эту ложь я повторяла в последнее время так часто, что она уже звучала в моих устах почти убедительно. 

Неточности/вольности в передаче содержания - передача ключевой информации без учета формальных компонентов исходного текста.

Пример:  In the West cut diamonds outnumber cars --- На Западе бриллиантов больше чем сотовых телефонов.

Нарушение норм ПЯ (буквализмы, нарушение правил сочетаемости слов и пр.) 

Буквализм  ошибка переводчика, заключающаяся в передаче слова, словосочетания или фразы в ущерб смыслу или информации о структуре. 

Пример: --- It is a good horse that never stumbles. 

Ошибочный вариант: Это хорошая лошадь, которая никогда не спотыкается. 

Верный перевод:  Конь на четырех ногах, и тот спотыкается. (Не бывает такой лошади, которая хотя бы раз не споткнулась)

Нарушение жанрово-стилистических норм ПЯ --- немотивированное употребление  языковых  единиц,  не  соответствующих  общей  стилистической тональности текста. 

Невыполнение прагматической задачи перевода 


