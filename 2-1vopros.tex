\section{Теория перевода как самостоятельная наука}

Попытки осмыслить особенности и закономерности переводческой деятельности предпринимались на протяжении столетий. Анализируя переводческую практику, исследователи (прежде всего сами переводчики) формулировали определенные принципы перевода и рекомендации, которым должны следовать их коллеги. Однако подобные высказывания и          отдельные трактаты или переводческие комментарии не составляли последовательной теории перевода. Теория перевода, основанная на данных лингвистики (лингвистическая теория перевода), появилась только во второй половине XX века.

Становлению лингвистической теории перевода способствовало несколько объективных факторов. После окончания Второй мировой войны резко увеличился обмен информацией между людьми и народами. За так называемым «информационным взрывом» последовало значительное расширение масштабов переводческой деятельности во всем мире. Появились новые виды переводов: синхронный перевод, перевод (дублирование) кинофильмов, радиопередачи телепрограмм. В самой переводческой деятельности также произошли качественные изменения: если раньше переводились в основном лишь произведения художественной литературы, то теперь на первый план выдвинулись информативные переводы: научно-технические, общественно-политические, экономические, юридические и др. Все это не могло не вызвать огромной потребности в профессионально подготовленной, высококвалифицированной армии специалистов-переводчиков. Стало очевидным, что критерии оценки качества художественного перевода отличаются от тех, которые применяются в отношении информативных текстов, несопоставимы и условия работы переводчиков. Теперь уже недостаточно было готовить профессиональных переводчиков методом «индивидуального ученичества», когда какой-нибудь опытный переводчик опекал несколько учеников, знакомя их с тайнами переводческого мастерства. Во всех странах были открыты переводческие школы, факультеты, отделения университетов и тому подобные учебные заведения. Невиданные ранее масштабы переводческой деятельности и задачи массовой подготовки переводчиков сделали необходимым всестороннее изучение проблем перевода. Таким образом, стала очевидной необходимость теоретической научно-обоснованной обработки дотоле лишь интуитивных догадок и чисто эмпирических наблюдений отдельных переводчиков-практиков. Так было положено начало науке о переводе --- переводоведении.

В 1950--1960-х гг. публикуются работы отечественных ученых Я. И. Рецкера, А. В. Федорова, И. И. Ревзина и В. Ю. Розенцвейга, зарубежных исследователей Ю. Найды, Ж. Мунена, Дж. Кэтфорда. Постепенно происходит слом традиционного исключительно «филологического» взгляда на перевод, которое основывается на заблуждении, что знания иностранного языка достаточно, чтобы быть переводчиком.

Среди крупнейших отечественных ученых-переводоведов 1970-х --- 1990гг. следует упомянуть Л. С. Бархударова, В. Н. Комиссарова, Л. К. Латышева, Р. К. Миньяр-Белоручева, Я. И. Рецкера, Г.В.Чернова, А. Д. Швейцера, среди зарубежных Г. Егера, В. Вилса, О. Каде, А. Нойберта, П. Ньюмарка. Переводоведы обращаются теперь ко всем без исключения видам и подвидам переводческой деятельности, ко всем типам переводимых текстов, изучают деятельность       переводчика в различных условиях. Теория перевода постоянно пополняется новыми авторами и новыми трудами.

\subsection*{Объект, предмет, методы, цели и задачи теории перевода}

В качестве названия науки о переводе в современной литературе используются два термина --- переводоведение и теория перевода. Как правило, они рассматриваются как равнозначные, однако если автор желает подчеркнуть чисто языковой подход к исследованию проблем перевода, то используется термин лингвистическая теория перевода или просто теория перевода.  Термин переводоведение обычно указывает на то, что автор придерживается более широкого подхода к предмету науки о переводе, акцентируя многоаспектный и междисциплинарный характер исследований.

Объект теории перевода --- это непосредственный процесс переводческой деятельности, а также результат такой деятельности --- текст перевода.

Предметом теории перевода является изучение закономерностей переводческого процесса, а также факторов, влияющих на протекание этого процесса и определяющих результат процесса перевода.

Основным методом теории перевода является сопоставительный анализ перевода или переводов с оригиналом, самих переводов друг с другом.

Как всякая научная дисциплина, теория перевода имеет свою структуру, которая включает три ветви:

Общая теория перевода изучает наиболее общие закономерности перевода как интеллектуальной деятельности, независимо от конкретной пары языков. Ее задача заключается, прежде всего, в исследовании тех факторов и проблем, которые раскрывают основные положения перевода как деятельности.

Частная теория перевода рассматривает закономерные соответствия двух конкретных языков и регулярные способы перевода с одного данного языка на другой данный язык и наоборот. Так, например, можно говорить о частной теории перевода английского и русского языков.

Специальная теория перевода раскрывает особенности процесса перевода текстов разных типов и жанров (исследования в области научно-технического, военного, юридического перевода), а также изучает частные закономерности определенных видов переводческой деятельности (теория художественного перевода, теория устного перевода и т.п.).

Теория перевода ставит перед собой следующие задачи:
\\
1) раскрыть и описать общелингвистические основы перевода, т.е. указать какие особенности языковых систем и закономерности функционирования языков лежат в основе переводческого процесса, делают этот процесс возможным и определяют его характер и границы;
\\
2) определить перевод как объект лингвистического исследования, указать его отличие от других видов языкового посредничества;
\\
3) разработать основы классификации видов переводческой деятельности;
\\
4) раскрыть сущность переводческой эквивалентности как основы коммуникативной равноценности текстов оригинала и перевода;
\\
5) разработать общие принципы и особенности построения частных и специальных теорий перевода для различных комбинаций зыков:
\\
6) разработать общие принципы научного описания процесса перевода как действий переводчика по преобразованию текста оригинала в текст перевода;
\\
7) раскрыть воздействие на процесс перевода прагматических и социолингвистических факторов;
\\
8) определить понятие “норма перевода” и разработать принципы оценки качества перевода.

Научный анализ лингвистических механизмов перевода делает возможным выработку некоторых нормативных (прескриптивных) предписаний, принципов и правил, методов и приемов перевода, которые помогают переводчику принять решение на перевод.

Таким образом, теория перевода опосредованно способствует переводческой практике, которая могла бы черпать в ней выводы и доказательства для выбора нужных средств выражения и в пользу определенного решения конкретных задач.
