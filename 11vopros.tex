\section{Перевод реалий и неологизмов}

Реалии и неологизмы входят в состав безэквивалентной лексики. Под безэквивалентной лексикой имеются в виду лексические единицы (слова и устойчивые словосочетания) одного из языков, которые не имеют ни полных, ни частичных эквивалентов среди лексических единиц другого языка. (Л.С. Бархударов)

Неологизм  --- это слово, значение слова или словосочетание, недавно появившееся в языке (новообразованное, отсутствовавшее ранее).

Реалии (экзотизмы) --- слова, обозначающие предметы, понятия и ситуации, не существующие в практическом опыте людей, говорящих на другом языке. Сюда относятся слова, обозначающие разного рода предметы материальной и духовной культуры, свойственные только данному народу. 

Классификация реалий:

1. Бытовые реалии (жилище, одежда, пища, виды труда, денежные знаки, музыкальные инструменты, народные праздники), кимоно (Япония), лапти (Россия), изба (Россия), сауна (Финляндия), аршин (Россия), рубль (Россия), скальд (Исландия);

2. Этнографические и мифологические реалии: Дед Mopoз (Россия), тролль (Скандинавия), пагода (буддизм). мечеть (мусульманство);

3. Реалии природного мира (животные, растения, ландшафт), фиорд (Норвегия), прерия (Латинская Америка), саванна (Африка), эндемики: секвойя, баобаб, кенгуру;

4. Реалии государственного строя и общественной жизни (актуальные и исторические): виги и тори (Англия), большевики (Россия);

5. Ономастические реалии --- антропонимы (имена, фамилии известных личностей, требующие комментариев), топонимы, имена литературных героев других произведений, названия музеев;

6. Ассоциативные реалии --- вегетативные символы, анималистические символы, цветовая символика, фольклорные, исторические и литературно-книжные аллюзии, языковые аллюзии.

Л.С. Бархударов приводит следующие способы передачи безэквивалентной лексики:

1. Транслитерация или транскрипция. При транслитерации средствами ПЯ передается графическая форма (буквенный состав) слова ИЯ, а при транскрипции --- его звуковая форма. Эти способы применяются при передаче иноязычных имен собственных, географических наименований и названий разного рода компаний, фирм, пароходов, гостиниц, газет, журналов и пр. Hot dog --- хот дог, marketing --- маркетинг, know-how --- ноу-хау, London --- Лондон, “General Motors” --- «Дженерал Моторс», “The New York Times” --- Нью Йорк Таймс. Борщ, щи, квас, калач, рассольник, окрошка, самовар --- borshch, shchi, kvass, kalatch, rassolnik, okroshka, samovar, Пушкин, Достоевский --- Pushkin, Dostoevski. Facebooker --- фейсбукер. A blog is a journal available for other people to read on the web. Блог --- это журнал, который люди могут прочитать в Сети. 

2. Калькирование. Этот прием заключается в передаче безэквивалентной лексики ИЯ при помощи замены ее составных частей --- морфем или слов (в случае устойчивых словосочетании) их прямыми лексическими соответствиями в ПЯ.  Как транскрипция и транслитерация, так и калькирование не всегда раскрывает для читателя, незнакомого с ИЯ, значение переводимого слова или словосочетания. Thanksgiving Day --- День Благодарения, skyscraper --- небоскреб, hot line --- горячая линия, to make money --- делать деньги, shadow economy --- теневая экономика.

3. Описательный («разъяснительный») перевод. Этот способ передачи безэквивалентной лексики заключается в раскрытии значения лексической единицы ИЯ при помощи развернутых словосочетаний, раскрывающих существенные признаки обозначаемого данной лексической единицей явления, то есть, по сути дела, при помощи ее определения на ПЯ. Часто переводчики прибегают к сочетанию двух приемов --- транскрипции или калькирования и описательного перевода, давая последний в сноске. Щи --- cabbage soup, погорелец  --- a person who has lost all his possessions in a fire, eye-opener --- что-либо, открывающее человеку глаза на действительное положение вещей, Tutor --- руководитель группы студентов в английском университете. Redundancy --- увольнение по сокращению штатов.

4. Приближенный перевод (перевод при помощи «аналога») заключается в подыскании ближайшего по значению соответствия в ПЯ для лексической единицы ИЯ, не имеющей в ПЯ точных соответствий. Drugstore --- аптека, know-how ---  секреты производства, Santa Claus --- Дед Мороз,  Bigfoot --- снежный человек.

5. Трансформационный перевод. В ряде случаев при передаче безэквивалентной лексики приходится прибегать к перестройке синтаксической структуры предложения, к лексическим заменам с полным изменением значения исходного слова или же к тому и другому одновременно, то есть к тому, что носит название лексико-грамматических переводческих трансформаций. Пример: Предложение «Не died of exposure» - невозможно перевести на русский язык, не прибегая к широкому контексту, так как существительное the exposure не имеет прямых переводческих эквивалентов, его можно перевести как: Он умер от простуды (от воспаления легких), Он погиб от солнечного удара, Он замерз в снегах и т.д. 