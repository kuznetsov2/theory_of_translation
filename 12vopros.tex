\section{Профессиональная компетенция переводчика. Инструменты переводческой деятельности}

Согласно В.Н. Комисарову, переводческая компетенция включает в себя  языковую, текстообразующую, коммуникативную, техническую и личностную характеристики переводчика.

Языковая компетенция переводчика включает все аспекты владения языком, характерные для носителя языка, но, кроме того, подразумевает и ряд специфических особенностей. Переводчик должен помнить о системе, норме и узусе языка, о его словарном составе и грамматическом строе, о правилах использования единиц языка для построения речевых высказываний. Он должен обладать этой компетенцией как в рецептивном, так и в продуктивном планах в обоих языках, участвующих в процессе перевода.

Умение создавать тексты различного типа в соответствии с коммуникативной задачей и ситуацией общения, обеспечивать надлежащую структуру текста, использовать языковые единицы текста по правилам построения речевых единиц в языке, оценивать место и соотношение отдельных частей текста и воспринимать текст как связное речевое целое составляет текстообразующую компетенцию. В текстообразующую компетенцию переводчика помимо описанных выше умений также входят и знания различий в общей стратегии построения текста в двух языках.

Общение людей с помощью языка осуществляется своеобразным, сложным путем, и достаточное владение языком --- это лишь одно из условий коммуникации. Коммуниканты дополняют языковое содержание высказывания информацией, которую они извлекают из обстановки общения и предыдущего опыта и знаний о мире, т.е. фоновых знаний. Способность человека к интерференции --- формированию правильных выводов из речевых высказываний о их полном содержании на основе фоновых знаний --- составляет его коммуникативную компетенцию.

Техническая компетенция переводчика включает знания, умения и навыки, необходимые для выполнения переводческой деятельности. Прежде всего, это знания о стратегиях перевода, переводческих приемах и трансформациях.

Профессиональная компетенция переводчика включает и некоторые личностные характеристики. Перевод представляет собой сложный вид умственной деятельности, осуществление которой предполагает особую психическую организацию, гибкость, способность быстро переключать внимание, переходить от одного языка к другому, от одной культуры к другой.

Переводчику приходится переводить тексты различной тематики, поэтому от него требуется широта интересов, эрудированность и начитанность, умение постоянно обогащать знания, использовать различные справочники и другие источники информации. Важными для переводчика являются его морально-этические установки.

\subsection*{Инструменты переводческой деятельности}

CAT-инструменты --- это программное обеспечение, облегчающее труд письменных переводчиков. CAT расшифровывается как computer-aided (assisted) translation. Эти программы также называют программами translation memory (ТМ). Перевод, выполненный с помощью CAT-инструментов, --- это не машинный перевод. Программа лишь помогает человеку выполнить перевод быстрее, а также повысить его качество.

Исходные файлы для перевода конвертируются в специальный формат, обрабатываемый определенным CAT-инструментом. Переводчик открывает его с помощью этой же программы и приступает к работе. В процессе перевода исходная фраза и ее перевод записываются в базу данных (ТМ). Если далее в тексте попадется такая же фраза, программа автоматически подставит соответствующий перевод из ТМ. У переводчика есть возможность отредактировать этот подставленный перевод, если в этом есть необходимость. Если же в тексте попадается очень похожая фраза на ту, что уже внесена в ТМ, подставляется имеющийся перевод, а программа показывает, что и как изменилось (исчезло/появилось слово, слова поменяли местами и т. д.).

Преимущества:
\\
1. Увеличение скорости работы
\\
2. Повышение качества перевода
\\
3. Автоматическое использование глоссария

CAT-инструменты позволяют подключать готовые глоссарии, полученные от клиента, а также составлять собственные в процессе перевода. Нет необходимости запоминать, как в этом проекте переводится тот или иной термин. Нужно просто добавить его перевод в глоссарий. Как только в предложении встретится слово или фраза из глоссария, сразу же будет виден и перевод.

4. Наличие наработанной ТМ по проекту, серверной (удаленной) ТМ

Если над проектом работают сразу несколько человек, им можно предоставить доступ к единой удаленной ТМ и тем самым избежать разнобоя в терминологии и стиле.

5. Формальная проверка качества

Файлы, переведенные с помощью CAT-программ, можно проверить специальными (и часто бесплатными) программами в отношении: единообразия перевода (посегментно); соответствия глоссарию; орфографии; пропусков; непереведенных фрагментов вместо перевода; несоответствия цифр; двойных пробелов; несоответствия тегов; повторяющихся слов; парности скобок и кавычек; расстановки пробелов в сочетании со знаками препинания и другими символами и др.

Trados --- система автоматизированного перевода, первоначально (с 1992 года) разработанная немецкой компанией Trados GmbH. Является одним из мировых лидеров в классе систем Translation Memory (TM, Память переводов).

Система Trados состоит из модулей, предназначенных для перевода текстов различного формата: документов Microsoft Word, презентаций PowerPoint, текстов в формате HTML и других метаданных, а также для ведения терминологических баз данных (модуль MultiTerm).

Концепция Translation Memory предполагает выявление в переводимом тексте фрагментов, переводы которых уже имеются в базе данных переводов, и за счет этого сокращение объёма работы переводчика. Фрагменты, оставшиеся непереведёнными, передаются дальше для ручной обработки переводчику или системе машинного перевода (Machine Translation, MT). Переводчик на этом этапе может выделить вновь переведённые фрагменты и занести новые пары параллельных текстов на двух языках в базу данных. Такая схема наилучшим образом работает в случае однотипных текстов, где повторяемость словосочетаний достаточно высока, то есть в случае различного рода инструкций для пользователей, технических описаний и т. п.