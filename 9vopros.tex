\section{Переводческие трансформации. Грамматические трансформации}

Грамматические трансформации --- межъязыковые трансформации при переводе, затрагивающие область грамматики.

1. Перестановка --- это изменение расположения (порядка следования) языковых элементов в тексте перевода по сравнению с текстом подлинника.

A suburban train was derailed near London last night. Вчера вечером вблизи Лондона сошел с рельс пригородный поезд.

Причиной изменения порядка слова является «актуальное членение предложения». В русском языке порядок слов в структуре предложения определяется «коммуникативным членением предложения». Это связано с существованием РЕМЫ (неизвестная/новая информация) и ТЕМЫ (известная/старая информация). В конце предложения, как правило, ставится «новое», то есть слова, несущие в себе впервые сообщаемую в данном предложении информацию (в нашем примере, сошел с рельс пригородный поезд). Второстепенные же элементы --- обстоятельства, обозначающие время и место действия, - располагаются обычно в начале предложения.

A boy came into the room. В комнату вошел мальчик.

A match flared in the darkness. В темноте вспыхнула спичка.

A small kitten was playing on the floor. На полу играл маленький котенок.

2. Замена членов предложения.

Чаще всего замена членов предложения вызывается необходимостью передачи «коммуникативного членения» предложения, о котором речь шла выше.  Что касается английского языка, то в нём порядок слов в предложении, как было сказано, определяется, прежде всего, факторами синтаксическими, то есть функцией того или иного слова как члена предложения: подлежащее в подавляющем большинстве случаев предшествует сказуемому, дополнение же следует за сказуемым.

Самый обычный пример такого рода синтаксической перестройки --- замена английской пассивной конструкции русской активной, при которой английскому подлежащему в русском предложении соответствует дополнение, стоящее в начале предложения (как «данное»); подлежащим в русском предложении становится слово, соответствующее английскому дополнению с by или же подлежащее вообще отсутствует (так называемая «неопределенно-личная» конструкция); форма страдательного залога английского глагола заменяется формой действительного залога русского глагола:

Не was met by his sister. Его встретила сестра.

He was given money. Ему дали денег.

I was offered another post. Мне предложили новую должность.

Visitors are requested to leave their coats in the cloakroom. Посетителей просят оставлять верхнюю одежду в гардеробе.

Еще примеры:

The last week has seen an intensification of the diplomatic activity... На прошлой неделе наблюдалась активизация дипломатической деятельности...

The fog stopped the traffic. Из-за тумана остановилось движение транспорта.

3. Замена частей речи. Переводчик прибегает к ней, когда в ПЯ нет части речи или конструкции с соответствующим значением, когда этого требуют нормы сочетаемости ПЯ. Существительное часто переводится глаголом, прилагательное — существительным, наречием и т. п.

--- Существительное заменяется глаголом:

It is my hope that… --- Я надеюсь, что…

She is a good cook… --- Она хорошо готовит…

I am an early riser… --- Я рано встаю.

--- Прилагательное заменяется существительным:

Youthful joblessness… --- безработица среди молодежи

--- Прилагательное заменяется глаголом:

He is hopeful… --- Он надеется

В процессе перевода английские прилагательные в сравнительной степени такие как higher, lower, longer, shorter, better, и т.п. часто заменяются существительными «повышение», «понижение», «увеличение», «сокращение», «улучшение» и т.п. Например:

They demand higher wages and better living conditions --- Они требуют повышения заработной платы и улучшения жизненных условий.

They fight for shorter hours and longer holidays --- Они борются за сокращение рабочего дня и увеличение продолжительности отпуска.

The accusation was disproved editorially. Это обвинение было опровергнуто в передовой статье. В переводе наречие editorially передается существительным с прилагательным, т. к. в русском языке нет эквивалента английскому наречию.

Ben's illness was public knowledge. О болезни Бена знали все. Сочетание public knowledge не имеет аналога в русском языке. Поэтому существительное knowledge заменено глаголом; прилагательное public в силу его широкой семантики можно заменить местоимением все. Синтаксис предложения претерпевает изменения: подлежащее illness становится дополнением, составное именное сказуемое в переводе заменяется простым глагольным.

4. Членение и объединение предложений.

Членение: 1 предложение ИЯ $\rightarrow$ 2 предложения ПЯ

The annual surveys of the Labour Government were not discussed with the workers at any stage, but only with the employers. --- Ежегодные обзоры лейбористского правительства не обсуждались среди рабочих ни на каком этапе. Они обсуждались только с предпринимателями.

He woke one morning to find himself famous. Однажды утром он проснулся и обнаружил, что он знаменит.

Объединение: 2 предложения ИЯ $\rightarrow$ 1 предложение ПЯ.

That was а long time ago. It seemed like fifty years ago. Это было давно - казалось, что прошло лет пятьдесят.

The only thing that worried me was our front door. It creaks like a bastard. Одно меня беспокоило --- наша парадная дверь скрипит как оголтелая.