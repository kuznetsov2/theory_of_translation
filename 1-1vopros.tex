\section{Сущность и определение перевода. История перевода (II)}

В повседневном, непрофессиональном понимании дать определение переводу достаточно просто. Любой случай, когда текст, созданный на одном языке перевыражается средствами другого языка мы называем переводом. Поскольку, язык - это своего рода код или знаковая система, т.е. произвольное обозначение предметов и явлений действительности с помощью условных знаков, то перевод можно назвать перекодированием, поскольку каждый из условных знаков заменяется при переводе знаком другой знаковой системы. Следовательно, перевод --- это перевыражение или перекодирование исходного текста. Другими словами, перевод представляет собой перевыражение исходного текста средствами другого языка. При этом термин «текст» понимается предельно широко: имеется в виду любое устное высказывание и любое письменное произведение от инструкции к телевизору до романа. Однако есть и ограничение: говоря о переводе, мы ограничиваемся вербальными текстами на живых человеческих языках. Понятие текста играет в переводоведении ключевую роль: переводчик, в отличие от лингвиста, имеет дело не столько с языком, сколько с его конкретными речевыми проявлениями --- текстами, поэтому сфера перевода в этом смысле --- царство речи. Под термином «текст» мы будем понимать речевое произведение, с помощью которого осуществляется вербальная коммуникация между людьми. Чаще всего отдельная законченная мысль реализуется в виде высказывания.

Всякий текст, как произведение речи, имеет определенное содержание, т.е. несет некую информацию, которая и подлежит передаче в процессе перевода. Это содержание иногда называют смыслом. Смысл высказывания и значения составляющих его слов не тождественны друг другу. Значение относится к единицам языка, оно существует и тогда, когда единицы языка не употреблены в коммуникации и не «актуализировались» в речи, образовав тем самым смысл высказывания или текста. Переводчик оперирует языковыми единицами, но объектом передачи в переводе является именно смысл, а не слова. Иначе говоря, переводчик передает смысл всего текста, а не переводит отдельные слова. Этот основополагающий принцип перевода был высказан христианским писателем и богословом Иеронимом Стридонским еще в IV веке н.э.: «…я передаю не слово словом, а мысль мыслью».

Несоблюдение этого ключевого принципа перевода часто приводит к буквальному переводу, т.е. переводу «слово в слово». В результате нарушаются не только нормы языка, но и искажается смысл оригинала. Наглядной иллюстрацией «ловушек» буквального перевода могут служить пословицы, поговорки, разговорные клише, в которых общий смысл высказывания совсем не столь очевиден, как это может показаться на первый взгляд.

Итак, универсальная задача перевода заключается в том, чтобы максимально полно передать информацию, заложенную в оригинале. На характер переводческой деятельности откладывает отпечаток и сам тип передаваемой информации. Различают следующие основные типы информации:
\\
--- познавательная (когнитивная, фактическая) информация --- объективные сведения об окружающем мире. Передавая ее, язык выполняет когнитивную или денотативную функцию, т.е. представляет предметы и явления окружающей действительности.
\\
--- эмоциональная (экспрессивная) информация --- содержит сообщение о человеческих эмоциях. Предавая эмоции, язык выполняет экспрессивную функцию.
\\
--- оперативная информация --- предписывает определенные действия или побуждает к ним, язык при этом выполняет побудительную функцию.

Характер информации в огромной степени зависит от типа текста. Например, в официальных документах и научной литературе доминирует информация когнитивная (фактическая). В художественных, особенно в поэтических, произведениях очень важна информация эмоционально-экспрессивная, создаваемая особым отбором выразительных средств, рассчитанных на определенное воздействие на читателя. В газетно-публицистических материалах, как правило, сочетаются оба вида информации, и удельный вес каждого из них зависит от тематики, авторской манеры, характера печатного издания и т.д.

Ранее мы определили процесс перевода как перевыражение или перекодирование исходного текста. Однако это перекодирование не является природным процессом, его осуществляет человек. Человек обладает индивидуальностью и способностью к творчеству. Именно эти два фактора позволяют ему при перекодировании выбрать из нескольких вариантов перевода свой. Поэтому, говоря о творческом или эвристическом характере процесса перевода, прежде всего, понимают свободу выбора. Возможность для выбора дает переводчику сам язык или т.н. языковая избыточность, т.е. наличие избыточных, запасных вариантов кодирования. Широко известно, что одна и та же мысль, благодаря наличию синонимов, может быть выражена разными способами, и это в полной мере относится и к переводу.

Но свобода выбора переводчика не абсолютна. Переводчик не может предложить при перекодировании абсолютно любой вариант, который ему понравился. Свобода выбора для переводчика в первую очередь ограничивается ресурсами языка т.е. тем возможны ли в нем принципе варианты. Так, при переводе однозначных соответствий (терминов, географических названий) у переводчика практически нет выбора. Накладывает ограничения на переводчика и грамматика, например, в английском языке принят фиксированный порядок слов, поэтому подлежащее почти всегда ставится в начало предложения. В русском языке новая информация (рема) всегда помещается в конец высказывания и переводчик просто вынужден соблюдать это правило.

Во-вторых, границы выбора могут быть разными в зависимости от вида и задач перевода. В письменном переводе требования при подборе соответствий более высокие; в устном, в обстановке дефицита времени, часто достаточно любого пришедшего на ум варианта. Кроме того, специфика устного перевода заставляет делать не полный, а сокращенный перевод, опускать некоторые смысловые компоненты, упрощать синтаксическую структуру.

В-третьих, границы свобода выбора может устанавливать тип текста. Если переводится стихотворный текст, то обязательно соответствие должно быть найдено для рифм, причем не обязательно с теми же словами, а вот вариантов замены каждого слова и вариантов построения всего текста становится много больше. При переводе юридических текстов от переводчика требуется максимальная точность и полнота передачи содержания, а значит нельзя упустить ни одно слово оригинала, включая, конечно, и термины, которым должны быть найдены однозначные соответствия на языке перевода. Структура высказываний и самого текста юридического документа также должна быть максимально приближена к оригиналу, что не дает переводчику большой свободы для выбора.

Наконец, в деятельности переводчика участвует фактор, не зависящий от внутриязыковых и внешних ситуативных причин. Это индивидуальность самого переводчика. Именно она придает особый, неповторимый оттенок творческому поиску вариантов. Переводчик может питать пристрастие к определенным словам, субъективно воспринимать специфику текста, у разных переводчиков разный словарный запас и опыт работы, что не может не сказываться на выборе вариантов. 

Рассмотренные нами параметры понятия «перевод» касались преимущественно описания его как процесса, но в ходе обсуждения стало очевидно, что тем же термином мы обозначаем и результат этого процесса. Таким образом, слово «перевод» соотносится с двумя различными понятиями: 1) перевод как некая интеллектуальная деятельность, и 2) перевод как результат этого процесса, т.е. текст, созданный переводчиком. На основании вышесказанного можно дать следующее определение перевода.

Перевод --- это деятельность, которая заключается в вариативном перевыражении, перекодировании текста, порожденного на одном языке в текст на другом языке, осуществляемая переводчиком, который творчески выбирает вариант в зависимости от вариативных ресурсов языка, вида перевода, задач перевода, типа текста и под воздействием собственной индивидуальности; перевод --- это также и результат этой деятельности.

\subsection*{История перевода}

Перевод, как и любая другая деятельность или профессия, возник из общественной потребности. Такой потребностью выступает необходимость в общении и обмене духовными ценностями между народами, говорящими на разных языках. Роль языка как средства общения и различие языков обусловили потребность в переводе и побудили людей к этому нелегкому, но столь необходимому труду.

Нет сомнения, что первые контакты между народами, говорящими на разных языках, были устными. Перевод, очевидно, долго существовал без письменной фиксации. Хотя у нас нет свидетельств о переводе дописьменной поры, можно с уверенностью предположить, что устный перевод появился задолго до возникновения письменности. Вполне возможно, что первыми переводчиками были женщины: выходя замуж за мужчину из другого племени, женщина овладевала языком мужа и могла впоследствии служить посредником при контактах между представителями разных племен.

Одно из первых упоминаний о переводческой деятельности восходит к XXIII в. до н.э. В городищах Месопотамии, на месте существования Древнего Шумера были найдены своеобразные шумеро-аккадские двуязычные словари --- глиняные таблички с начертанными на них списками слов на двух или трех языках. Во времена шумеро-аккадской цивилизации переводились тесты преимущественно религиозного, административно-государственного и научного характера. В развитии переводческого дела в Шумере большую роль играли школы, в которых обучали писцов. Поскольку канцелярских языков было два, каждый писец обязательно должен был уметь переводить.

Известно, что и в других цивилизациях Древнего Мира (Вавилон, Ассирия, Карфаген, Древний Египет) существовали специальные группы, или профессиональные касты переводчиков, доводивших повеления властителей до сведения покоренных народов.

Примечательно, что имена переводчиков древности практически не сохранились, а первый        переводчик, которого мы знаем по имени - Анхурмес, был египетским верховным жрецом в XIV в. до н.э.

 Перевод в эпоху античности. Две основные функции перевода

В Древней Греции перевод с других языков был развит слабо. Древнегреческая цивилизация мало подпитывалась воздействием извне и ощущала себя как самодостаточная, и необходимости в переводе произведений с чужих языков не возникало. Известно, что греки с высокомерием относились к другим народам, называя их варварами, и неохотно изучали «варварские» языки.

Древние римляне, в отличие от древних греков, переводом пользовались широко, причем преобладал перевод с греческого языка, сыгравший значительную роль в формировании римской культуры периода античности. Перевод осознавался как действенный способ обогащения культуры молодого государства.

Родоначальником перевода письменных памятников и зачинателем римской литературы считается Луций Ливий Андроник (ок. 275-200гг. до н.э.). Андроник перевел на латинский язык «Одиссею» Гомера, разработал различные приемы адаптации греческих произведений к римской действительности и культуре. Именно он первым заменял при переводе имена греческих богов на имена соответствующих им римских богов. С помощью транскрипции он вводил в текст латинского перевода греческие слова, обозначавшие экзотические реалии.

К другим знаменитым переводчикам античности относятся Теренций, Цицерон, Гораций, Вергилий, Боэций. Поистине, грандиозно переводческое творчество Теренция (190--159гг. до н.э.), который перевел с греческого около 100 комедий. Перу другого древнегреческого переводчика Боэция принадлежат переводы философских сочинений Пифагора, Птолемея, Евклида, Платона, Аристотеля. Со времен античности процесс вывода переводчика и его профессии из тени, пусть и не всегда успешно, но все же неотвратимо набирает силу.

Первые теоретические соображения о переводе принадлежат Цицерону (106--43гг. до н.э.). Цицерон считал, что буквальный перевод --- свидетельство языковой бедности и беспомощности переводчика. Он призывал передавать не форму, а смысл произведения, следовать законам языка перевода, ориентироваться при выборе соответствий на читателя или слушателя.

Перевод в эпоху Средневековья. Буквальный перевод

Закат античности и начало новой эры в развитии европейской цивилизации ознаменовался распространением новой монотеистической религии --- христианства. Христианство принесло с собой Священное писание --- священный текст, к которому люди уже не могли подходить с прежними, античными мерками. Текст этот, данный Богом и имеющий законченное письменное оформление, почитался как святыня. Почитание текста Священного писания основывалось на почитании Слова, которое воспринималось как наименьшая возможная частица, связывающая человека с Богом. Такое иконическое представление о природе слова предполагало, что слово есть образ вещи, причем существует внутренняя нерасторжимая связь между словом и вещью. Это средневековое европейское восприятие текста дает ключ и к пониманию средневековой теории перевода. Если слово есть образ вещи, то оно может отобразить вещь еще раз --- на другом языке. Для этого переводчик должен был выбрать другой иконический знак для данного слова, вернее --- тот же по сути, но выраженный средствами другого языка. Но в таком случае переводчик обязан четко и последовательно, без изъятий, отображать каждое слово, иначе он исказит реальность. Тот факт, что при этом перевод часто становился темным и непонятным никого не смущало, ни самих переводчиков, ни читателей, которые полагали, что религиозный текст и не может быть понятен каждому, в нем многое должно быть мистическим, загадочным. Со времен Средневековья пословный перевод, который нарушает нормы переводящего языка, а порой и искажает смысл подлинника по традиции называется буквальным переводом. В другие исторические эпохи переводчика-буквалиста за свою слепую приверженность тексту оригинала стали называть «рабом», а сам перевод --- «рабским». На самом деле, как уже отмечалось выше, такой подход к переводу был обусловлен философскими и религиозными представлениями той эпохи, а отнюдь не примитивизмом средневековых переводчиков.

Перевод в эпоху Возрождения. Принципы перевода Этьена Доле и Мартина Лютера

Новое поколение светских переводчиков того времени единодушно высказываются в пользу перевода, точно передающего подлинник по смыслу и соблюдающего нормы родного языка.

Известный французский гуманист и переводчик Этьен Доле одним из первых выразил новые представления о переводе в своем трактате «О способе хорошо переводить с одного языка на другой». Так, Этьен Доле считал, что переводчик должен соблюдать следующие пять основных принципов перевода:
\\
1) он должен в совершенстве понимать содержание переводимого текста и намерение автора, которого переводит;
\\
2) он должен в совершенстве владеть языком, с которого он переводит, и столь же превосходно знать язык, на который переводит;
\\
3) он должен избегать тенденции переводить слово в слово, ибо это исказило бы содержание оригинала и погубило бы красоту его формы;
\\
4) он должен использовать в переводе общеупотребительные формы речи;
\\
5) правильно выбирая и располагая слова, он должен передать общее впечатление, производимое оригиналом в соответствующей «тональности».

Церковь в эпоху возрождения ужесточает борьбу за свою чистоту, которую она также связывает с буквальным переводом. Пострадал и Этьен Доле. За неканоническое истолкование одной реплики Сократа он был приговорен церковным судом к смерти и сожжен на костре в 1546 г.

Однако настоящий перелом в истории перевода наступил только тогда, когда ревизии подвергается главный текст в жизни людей того времени --- Библия. Этот перелом связан с реформацией самой церкви и в первую очередь с именем Мартина Лютера (1483--1546). Мартин Лютер, немецкий священник, доктор теологии предложил принципиально новый перевод Священного писания, который с одной стороны соблюдал каноническую полноту и точность передачи содержания, а с другой был понятен и знаком любому человеку, т.е. ориентировался на нормы общенародного языка. Именно такой текст должен был позволить простому народу общаться с Богом без посредников. Новый перевод Библии, выполненный на основе принципов Мартина Лютера, стал основной опорой Реформации христианской церкви.

Классицистический период. Вольный перевод

Начиная с конца XVII в. в европейских литературах определяются принципы перевода, согласно которым текст должен отвечать нормам эстетики классицизма. Лучшим переводом признавался перевод, приближенный к некоему художественному идеалу. Перевод должен был удовлетворять требованиям «хорошего вкуса», поэтому самый приятный перевод считался и самым верным. Текст оригинала рассматривался как сырой, несовершенный материал, поэтому вполне допустимыми считались как изменение содержания (сюжета, композиции, состава персонажей), так и средств выражения этого содержания. В результате исчезало всякое авторское и стилистическое своеобразие подлинника. Своеобразным было также и отношение к нормам языка перевода. Текст должен был соответствовать нормам того языка, на который переводился, но эти нормы трактовались иначе, чем в концепции Лютера. Установка на общенародный язык сменилась на язык литературного жанра, во многом далекий от разговорного. Так, при переводе трагедий Шекспира на французский язык устранялись все отступления от эстетического идеала, каковыми считались смешение прозы и поэзии, употребление грубой и просторечной лексики и др. Такое свободное обращение с текстом оригинала является примером вольного перевода, в котором переводчик по каким-либо причинам (в силу низкой квалификации или отсутствия должного уважения к переводимому тексту) допускает неоправданные отклонения от подлинника или его искажение. Другой причиной вольного перевода может быть чрезмерное и необоснованное использование переводчиком приемов культурной адаптации, в результате чего переводимый текст либо полностью, либо частично теряет национальную специфику и авторский стиль. В качестве примера можно привести переводческую практику, которая сложилась в России в конце XVII --- начале XIX века и получила название «склонение на наши нравы». Так, у Гавриила Романовича Державина в переводе Горация «Похвала сельской жизни» читатель встречает упоминание о «горшке горячих добрых щей» и сугубо русскую реалию «Петров день».

Романтический перевод. Теория непереводимости

В конце XVIII в. отчетливо проявляются симптомы нового отношения к тексту и его переводу. Изменение отношения к оригиналу позже очень точно сформулировал Пушкин: «От переводчиков стали требовать более верности и менее щекотливости и усердия к публике --- пожелали видеть Данте, Шекспира и Сервантеса в их собственном виде, в их народной одежде». 

Отмеченные различия в языках разных народов, которые увязывались воедино с их историей и культурой, дали толчок сравнительным лингвистическим исследованиям. Появилось сравнительное языкознание. Каждый народ, согласно рассуждениям одного из основоположников сравнительного языкознания Вильгельма Гумбольдта, мыслит и чувствует по-разному, что отражается в его языке; язык же, в свою очередь, воздействует на человека активно. Получается такая специфика, которую вряд ли можно передать средствами другого языка. В результате таких рассуждений в конце XVIII в. впервые зарождается сомнение в возможности перевода. Наиболее категорично это сомнение формулирует именно Вильгельм Гумбольдт: «Всякий перевод безусловно представляется мне попыткой разрешить невыполнимую задачу. Ибо каждый переводчик неизбежно должен разбиться об один из двух подводных камней, слишком точно придерживаясь либо своего подлинника за счет вкуса и языка собственного народа, либо своеобразия собственного народа за счет своего подлинника. Нечто среднее между тем и другим не только трудно достижимо, но и просто невозможно». Следует отметить, что теория непереводимости не получила широкого распространения, поскольку противоречила самой практике перевода.

«Золотой век» перевода в России

В XIX веке романтизм пришел и в Россию. Это столетие по праву называют «золотым веком» не только русской литературы, но и русского перевода. Новая русская школа перевода начала формироваться благодаря, в первую очередь, вкладу таких известных деятелей культуры, как историк А. Карамзин и поэт В. Жуковский. Последнего Пушкин называл «гением перевода».  Жуковский был талантливым поэтом, но значительную часть его творчества составляли переводы. Он переводил с английского, французского, старославянского, латинского и немецкого языков. Благодаря ему русские читатели получили доступ ко многим произведениям Шиллера, Гете, Байрона, Вальтера Скотта и других корифеев мировой литературы. Диапазон его творческих поисков поразителен: от переводов сказок Шарля Перро и братьев Гримм до «Одиссеи» Гомера и знаменитого эпоса «Слово о полку Игореве». Почетное место в истории перевода в России принадлежит двум великим русским поэтам А. С. Пушкину и М. Ю. Лермонтову, чьи переводы-парафразы послужили образцовыми примерами для других переводчиков.