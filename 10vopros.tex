\section{Интернациональная лексика, «ложные друзья переводчика»}

К интернациональной лексике обычно относятся слова, имеющие в результате взаимовлияний или случайных совпадений внешне сходную форму и некоторые одинаковые значения в разных языках. Интернациональные слова попадают в тот или иной язык либо благодаря заимствованию из другого языка, либо вследствие того, что два данных языка заимствовали соответствующее слово из какого-нибудь третьего языка (например, из латинского или греческого). Так, слова accumulator, alpha, alphabet, ampere, atlas, atom, banjo, barbarism, benzol, billiards, bulldog, cafeteria, caravan, catastrophe, chaos, demagogue, dynamo, economic, electric, element, energy, film, kodak, legal, minimum, philosopher являются общепонятными не только для лиц - носителей английского языка, но и для тех, у кого родной язык другой. Это достигается за счет интернационального характера указанных слов. 

Вместе с тем, слова, ассоциируемые и отождествляемые (благодаря сходству в плане выражения) в двух языках, в плане содержания или по употреблению не полностью соответствуют или даже полностью не соответствуют друг другу. Именно поэтому слова такого типа получили во французском языкознании название faux amis du traducteur --- «ложных друзей переводчика».

Допустимость буквального перевода терминов и терминологических сочетаний в ряде отдельных редких случаев может послужить источником ошибок в переводе. Как отмечает исследователь научно-технического перевода А. Л. Пумпянский, к основным причинам, приводящим к ошибкам, относятся:\\
 1) убежденность в однозначности слов и грамматических форм; \\
 2) смешение графического облика слова; \\
 3) ошибочное использование аналогии;\\
 4) перевод слов более конкретными значениями, чем они фактически имеют;\\
 5) неумение подыскать русское значение для перевода английских слов и лексических и грамматических сочетаний;\\
 6) незнание закономерностей изложения английского научно-технического материала и способа его передачи на русский язык».

В литературе отмечаются следующие возможные расхождения значений интернациональных и соответствующих русских слов: 

1. Русское слово совпадает с английским, но не во всех значениях, а лишь в одном - двух. В эту группу интернациональной лексики входит сравнительно большое количество слов, перевод которых представляет существенные трудности.

Перешедшее в русский язык из английского языка многозначное слово сателлит, употребляется главным образом лишь в одном значении: 'государство-сателлит', 'марионеточное государство'. В английском же языке слово satellite реализуется в нескольких значениях: 'сателлит', 'спутник', 'искусственный спутник', 'член свиты', 'участник торжественного кортежа', 'сопровождающее лицо', 'приспешник', 'приверженец', государство-сателлит', 'город-спутник', 'спутник' (хромосомы).

2. Русское слово имеет ряд значений и лишь одно из них соответствует английскому. Обычно это наблюдается тогда, когда слово заимствовано из какого-либо третьего языка: так, русское слово аудитория шире по значению английского auditorium. По-русски можно сказать аудитория читателей; по-английски слово auditorium в таком значении не употребляется, и эквивалентами в английском языке для передачи этого значения будут такие единицы, как the readership, the reading audience, the readers или даже market (the book has a good market).

Некоторые сходные по форме слова нередко имеют различные основные значения. Так, activities переводится как деятельность, а не как активность; communal в основном употребляется в значении 'общественный' и очень редко --- 'коммунальный'; aspirant ---претендент на что-либо, но никогда не аспирант; direction - направление, а не дирекция (board of directors; management); magazine-журнал, но, не магазин; obligation-обязательство, а не облигация; sympathetic-сочувствующий, но не --- симпатичный (sympathetic strike - забастовка солидарности, а не забастовка симпатии); fabric - текстильное изделие, а не фабрика.

Перенесение близких по звучанию (или имеющих сходную графическую форму) слов из одного языка в другой при переводе особенно часто наблюдается в родственных языках, например в русском и украинском. Однако, при перенесении такого слова из одного языка в другой, могут потеряться такие качества как народность или книжность слова, эмоциональность или терминологичность, позитивность или негативность значения, поэтичность, торжественность и так далее - до тончайших оттенков в бесконечных вариациях и соотношениях их.(О. Кундзич)

Недоучет различий, о которых пишет О. Кундзич, часто наблюдается также, при переводе с русского языка на английский. Вот некоторые типичные примеры неправильного перевода.

Слово или словосочетание --- Неточный перевод (влияние слов, имеющих сходную графическую форму с русскими словами) --- Терминологичный адекватный перевод
\\

фальшивые (подделанные) документы --- false documents --- forged documents (papers)
\\

сплошная фикция --- all fiction --- pure invention
\\

газифицировать (деревню) --- to gas, to gassify (a village) --- to provide gas, gas supply
\\

гениальное изобретение --- genial invention --- great invention
\\

плотничий инструмент --- carpenter's instrument --- carpenter's tools
\\

романист (о писателе) --- romanist --- novelist
\\

секретная информация --- secret intormation --- classified information
\\

центральная газета --- central newspaper --- leading paper
\\

бесцеремонная манера --- unceremonial manner --- off-hand manner

«Ложные друзья переводчика», констатирует В. В. Акуленко, вводят в заблуждение не только начинающих переводчиков, но и опытных мастеров. При переводе с родного языка на иностранный проблема «ложных друзей переводчика» получает особое преломление. Забывая о том, что у сходного по форме английского слова может быть не одно, а несколько значений (причем в последних оно может использоваться в речи даже чаще, чем в первом), переводчики часто игнорируют второстепенные значения таких слов.

Другими словами, для лиц, изучающих иностранный язык, второстепенные значения слова гораздо труднее поддаются запоминанию, а тем более использованию в речи или переводе (что самым непосредственным образом сказывается на качестве перевода), тогда как носители иностранного языка в нужный момент «не забывают» об этих значениях и мгновенно извлекают их из своей памяти.

Данное положение не является чем-то новым в лингвистике или литературоведении. О. Кундзич пишет, что особенно часто переводчики и редакторы упускают из виду степень употребляемости слова в одном и другом языке, а это имеет очень большое значение. Опытный переводчик, как правило, решает такие вопросы интуитивно. Однозначная смысловая связь одного слова с другим образуется не только при оперировании лексикой, относимой к категории «ложных друзей переводчика». А. Д. Швейцер называет такое явление «ложной аналогией». Это первая причина нарушения стилистической нормы речи, а следовательно, и плохого качества перевода. Данная особенность речи тех, для кого английский язык не является родным, больше всего бросается в глаза носителям английского языка.

В какой-то степени связанной с рассматриваемым вопросом является проблема паронимов. Паронимы - однокоренные слова с частичным звуковым и структурным сходством. Неразличение паронимов в языке ведет к смешению этих слов, к смысловому искажению высказывания. Например, слово humanity (an act of humanity; to treat smb. with humanity) переводится на русский язык словом гуманность, но не гуманизм. Economical переводится на русский язык как экономичный (economical car), но не экономический (economic system). 
