\section{Теория перевода как самостоятельная наука. Объект, предмет, методы, цели и задачи теории перевода}

В 70-е годы ХХ века теория перевода начинает оформляется в самостоятельную научную дисциплину.

Самостоятельность любой науки определяется тем, что она имеет четко определенные объект и предмет исследования, сложившуюся понятийную систему. 

Современное состояние теории перевода характеризуется не только непрекращающимся поиском закономерностей переводческой деятельности, но и постоянным уточнением не только предмета, но и самого объекта этой науки.

%В плане общей теории познания противопоставление предмета и объекта является относительным. Основное структурное отличие предмета от объекта состоит в том, что предмет заключает в себе лишь главные, наиболее существенные с точки зрения конкретного исследования свойства и признаки.

А.В. Федоров, один из основоположников отечественной лингвистической теории перевода, справедливо полагал, что <<при всей взаимосвязанности различных плоскостей изучения, обусловленной единством самого объекта --- перевода, постоянно возникает необходимость обращать основное внимание на определенную сторону объекта изучения, при большей или меньшей степени абстракции от остальных (что естественно в науке)>>. Признавая, что на современном этапе интересы исследователей перевода разделились, что наряду с традиционным изучением соотношения перевода с оригиналом, появилось новое направление, исследующее процесс перевода путем его моделирования, Федоров считал необходимым особое внимание все же обращать на лингвистическую сторону перевода. <<Основным предметом внимания для теории перевода, --- отмечал он, --- являются соотношения между подлинником и переводом и различие тех форм, которые они принимают в конкретных случаях, требующих объяснения и уточнения>>.

А.Д. Швейцер, представляя иное направление в теории перевода, полагал, что в предмет теории перевода <<входит процесс перевода в широком социокультурном контексте с учетом влияющих на него внеязыковых факторов --- его социальных, культурных и психологических детерминантов>>. При этом объектом исследования оказывался опять же перевод как особый вид речевой коммуникации. Швейцер уже стремится преодолеть рамки лингвистики и вывести теорию перевода на уровень самостоятельной научной дисциплины.

Объект науки о переводе

Миньяр-Белоручев, определяя место теории перевода среди других отраслей знаний, прежде всего, постарался уточнить сам объект этой науки. От общего понятия <<перевод>> в концепции Федорова, через понятие <<особого вида коммуникации>> у Швейцера, Миньяр-Белоручев приходит к определению объекта науки о переводе как науки об особом виде коммуникации, а именно, коммуникации с использованием двух языков. Поэтому, считал он, теории перевода следует не ограничиваться сугубо лингвистическим аспектом процесса перевода, но также изучать и <<условия порождения исходного текста, и условия восприятия переводного текста, и социальный статус коммуникантов, и речевую ситуацию, и различные сопутствующие явления, что входит в сложное понятие коммуникации с использованием двух языков>>.

Предложенное исследователем положение об объекте переводческой науки представляет интерес, так как позволяет взглянуть на перевод не только как на объект сопоставительной лингвистики. Из этого определения мы узнаем, что перевод представляет собой один из сложных видов речевой деятельности и что он предполагает взаимодействие трех участников акта коммуникации с использованием двух языков. В этой формулировке объекта науки о переводе, %отражающей деятельностный подход к переводу, который, по мнению ее автора, противопоставлен сугубо лингвистическому, 
перевод рассматривается как процесс, рациональная деятельность.

Иначе говоря, наука о переводе исследует двуязычную деятельность переводчика в его взаимодействии с автором исходного сообщения и получателем продукта перевода. И, тем не менее, предложенное Миньяр-Белоручевым определение объекта теории перевода представляется, достаточно спорным, так как не отражает всех тех признаков, которыми может характеризоваться перевод. Определение предмета науки о переводе как теоретического отражения реальной переводческой практики также представляется расплывчатым.

Отождествление объекта и предмета теории перевода фактически происходит и у В.Н. Комиссарова. Этот исследователь, также как и Миньяр-Белоручев, рассматривает перевод в рамках межъязыковой коммуникации. Такой подход позволяет, по его мнению, <<решить вопрос о том, что составляет предмет теории перевода. Понятно, --- продолжает он, --- что теория перевода должна заниматься изучением перевода, но что такое перевод?>>.

Таким образом, предметом науки о переводе оказывается изучение объекта --- перевода, то есть всеобъемлющее теоретическое представление объекта без какой бы то ни было определенно выраженной специфики. В этом случае оказывается необходимым уточнить само понятие объекта, что и предпринимает Комиссаров, предлагая следующее определение перевода: <<Перевод --- это вид языкового посредничества, при котором на другом языке создается текст, предназначенный для полноправной замены оригинала, в качестве коммуникативно равнозначного последнему>>. Весьма лаконичное определение перевода, предложенное Комиссаровым, все же не отражает всей сущности объекта теории перевода с достаточной полнотой и ясностью. Комиссаров справедливо относит свое определение к разряду телеологических, то есть тех, которые пытаются объяснить сущность явлений и процессов через отношение целесообразности. Но, как известно, цель не является единственной причиной того, что процессы развиваются тем или иным способом. Более того, цель представляет собой лишь один из внешних факторов по отношению к процессу и не дает возможности вскрыть его сущность.

Попытка уточнить объект теории перевода вновь обращает нас к самому понятию <<перевод>>, ведь именно это понятие и должно отражать некую реальность, исследуемую данной научной дисциплиной.

Предмет теории перевода

При определении предмета науки о переводе следует исходить из того, что предметом любой науки является теоретическое отражение содержания объекта, создание его абстрактной модели, накопление систематизированных знаний об объекте.

Прежде чем попытаться определить предмет теории перевода, следует уточнить, что наука о переводе, какую бы направленность она не имела, была и останется теоретической дисциплиной, так как вряд ли когда-нибудь сможет располагать достаточными экспериментальными данными о психофизических процессах, протекающих в голове переводчика. Вряд ли она сможет в полной мере изучить и физическое состояние переводчика в процессе перевода. Можем ли мы представить себе переводчика, обеспечивающего переговоры на уровне глав государств, увешенного медицинскими датчиками, регистрирующими все изменения функционирования всех его жизненно важных органов, или синхрониста в кабине с аппаратом, измеряющим давление? Даже представить себе такое весьма сложно, реальные же эксперименты подобного рода вовсе невозможны. Мы можем лишь сопоставить физическое состояние переводчика до начала перевода и после. Далее, по результатам сравнения мы сможем зарегистрировать произошедшие в его физическом состоянии изменения. Эти изменения позволят, возможно, смоделировать процессы, происходящие во время перевода в организме переводчика. Повторим этот эксперимент множество раз и попытаемся вывести закономерности. Сделанные выводы лягут в основу дальнейших исследований.

Именно таким путем и развивается научное знание о переводе. Непосредственному наблюдению поддаются лишь материальные объекты и процессы, которые можно воспринять, проанализировать, измерить до начала процесса перевода и после его завершения, идет ли речь о психофизическом состоянии переводчика или о текстах --- исходном тексте и тексте перевода. В этом особенность перевода как объекта науки. Он позволяет оперировать лишь косвенными данными. Эта особенность объекта теории обусловливает и его предмет.

Тот факт, что реальному наблюдению и научному анализу поддаются лишь данные <<на входе>> и <<на выходе>>, в то время как сам интеллектуальный процесс переводческого преобразования происходит скрытно, превращает теорию перевода в сопоставительную дисциплину. Все выводы о механизме перевода делаются на основе сопоставления исходных и результирующих данных. Материальным продуктом, результирующим интеллектуальный процесс переводческого преобразования, оказывается речевое произведение. Поэтому естественно, что в качестве данных для сравнения выступают речевые произведения --- исходное, подвергшееся переводу, и финальное, созданное переводчиком.

Речевая сущность сравниваемых объектов и предопределила то, что предмет теории перевода долгое время практически не выходил за рамки предмета лингвистики. Это закономерно сводило теорию перевода к статусу прикладной отрасли языкознания. Теория перевода и называлась в этом случае лингвистической.

Л.С. Бархударов полагал, что <<предметом лингвистической теории перевода является научное описание процесса перевода как межъязыковой трансформации, то есть преобразования текста на одном языке в эквивалентный ему текст на другом языке>>. При этом он подчеркивал, что под термином <<процесс>> не имел в виду психический процесс, протекающий в мозгу переводчика во время перевода. О характере этого процесса имеются лишь весьма смутные представления, хотя этот процесс и представляет несомненный интерес.

Такое определение предмета теории перевода, ограничивающее его процессом межъязыковой трансформации, было вполне приемлемым для лингвистической теории перевода. Но оно показалось слишком узким, когда к изучению перевода начали подходить с позиций общей теории коммуникации. Р.К. Миньяр-Белоручев, полагал, что межъязыковые преобразования <<обязательно ограничены рамками двух конкретных языков… Тем самым задачи науки о переводе сводятся к сравнительному изучению двух языковых систем, к некоторому комплексу проблем частной теории перевода>>. Такой взгляд на предмет теории перевода ограничивается собственно лингвистическим аспектом перевода и не позволяет науке о переводе выйти за рамки лингвистики. По мнению этого исследователя, процесс перевода включен в коммуникацию с использованием двух языков и составляет ее центральное место. Всякое же моделирование коммуникации с использованием двух языков, накопленные о ней знания составляют предмет науки о переводе>>.

Таким образом, предмет науки о переводе полностью покрывает собой объект, становится своеобразной теоретической калькой объекта. Но, как мы пытались показать, перевод является чрезвычайно сложным объектом, предполагающим изучение разными научными дисциплинами, имеющими разные предметы, то есть междисциплинарный подход. Междисциплинарный подход вовсе не исключает того, что теория перевода, если, конечно, такая наука действительно существует как самостоятельная дисциплина, должна иметь собственный предмет. Представление о предмете науки о переводе, как о совокупности всех знаний о нем, накопленных разными научными дисциплинами, настолько широко раздвигает границы этой науки, что рискует лишить ее присущей каждой науке предметной определенности. Сравним перевод как объект науки с другим, чрезвычайно сложным, центральным, объектом научного знания в целом --- человеком. Человека как живое существо изучает биология, его болезни изучает медицина, его речь --- лингвистика, его поведение --- психология, отношения человека с ему подобными --- социология и т.п. Сложность и многосторонность объекта, бесчисленное множество вариантов его проявления делают абсурдной, по крайней мере, сегодня, идею создания единой <<науки о человеке>>, <<человековедения>>, <<гомологии>> или чего-то еще подобного. Перевод, несомненно, представляет собой менее сложный объект, нежели человек. Но, являясь одновременно фактом и посредничества, и речевой коммуникации, и билингвизма, и социального поведения, и психического состояния, и еще многого другого, перевод, тем не менее, имеет нечто специфическое, присущее только ему. Именно это специфическое и должно составить предмет теории перевода как самостоятельной научной дисциплины.

Поиск специфического вновь отправляет нас к реально существующим свидетельствам переводческой деятельности, а именно, речевым произведениям, текстам --- оригинальным, подлежащим переводу, и переводным. Явное отличие того, что имеется до перевода, от того, что рождается в результате перевода, со всей очевидностью показывает, что процесс перевода есть сложное, многообразное и многоуровневое преобразование, которое затрагивает самые различные аспекты речевой коммуникации. Преобразуется вся система смыслов исходного сообщения, что выражается в более или менее значительных преобразованиях семантики исходного сообщения и его прагматики, порядка взаимного расположения элементов сообщения, в изменении формы речи, например, письменной на устную при переводе с листа, художественной формы (поэтической на прозаическую и наоборот), литературного или речевого жанра и т.п. Все эти преобразования обусловлены самыми различными факторами самой различной природы, социальными, психическими, историческими, и др., что далеко выходит за пределы межъязыковой асимметрии. Именно поэтому взгляд на науку о переводе только с точки зрения лингвистики (лингвистическая теория перевода) представляется слишком узким и односторонним. В то же время попытка <<объять необъятное>> и представить предмет теории перевода как совокупность предметов различных научных дисциплин делает его неоправданно расширенным, размытым, а следовательно, и операционно непригодным. Попытка разрубить этот гордиев узел и совместить необходимую широту взгляда с точно обозначенной предметностью могла бы состоять в том, чтобы не прибавлять друг к другу, а синтезировать предметы разных наук, изучающих перевод, выделив собственный предмет. Ведь, именно таким путем выделяются новые научные направления. Но синтез различных предметов с целью построения единой теории объекта, способный привести к созданию новой научной дисциплины, требует системного подхода. Иначе говоря, наука о переводе должна синтезировать различные предметные стороны переводческой деятельности как некой системы.

В этом случае предметом теории перевода могла бы оказаться трансформирующая деятельность переводчика, создающего иное, тогда как целью его является создать нечто подобное. Именно это противоречие и порождающие его причины лингвистического, социального, психического, исторического, этнологического и др. плана и составляют предмет теории перевода. Противоречивость перевода, возникающая из столкновения <<своего>> и <<чужого>>, поиск подобия в различном и различия в подобном, можно объяснить, только синтезируя знания о переводе, накопленные разными науками. А.В. Федоров, намечая пути развития науки о переводе, справедливо полагал, что << дальнейший путь работы над теорией должен предполагать постепенно реализуемые возможности синтеза, путь от частного синтеза к синтезу более общему>>[15]. В качестве примера уже осуществившегося частного (но значительного по результату) синтеза Федоров называл <<преодоление существовавшего антагонизма между литературоведческим и лингвистическим путями изучения художественного перевода>>.

Синтез предметных сторон, имеющий целью построение единой теории перевода, требует системного подхода. Тогда сама противоречивая переводческая реальность будет представлена как системное явление, то есть как некая совокупность элементов, между которыми устанавливаются определенные типы связей и отношений, благодаря чему эта совокупность приобретает определенную целостность и единство, со всеми присущими системе свойствами и отношениями.

ТП ставит перед собой следующие основные задачи:

1. раскрыть и описать общелингвистические основы перевода, т.е. указать, какие особенности языковых систем и закономерности функционирования языков лежат в основе переводческого процесса, делают этот процесс возможным и опре­деляют его характер и границы;

2. определить перевод как объект лингвистического исследования, указать его отличие от других видов языкового по­средничества;

3. разработать основы классификации видов переводческой деятельности;

4. раскрыть сущность переводческой эквивалентности как основы коммуникативной равноценности текстов оригинала и перевода;

5. разработать общие принципы и особенности построения частных и специальных ТП для различных комбинаций языков;

6. разработать общие принципы научного описания процесса перевода как действий переводчика по преобразованию текста оригинала в текст перевода;

7. раскрыть воздействие на процесс перевода прагматических и социолингвистических факторов;

8. определить понятие «норма перевода» и разработать принципы оценки качества перевода.

ТП является лингвистической дисциплиной, поэтому широко использует данные и методы исследования других разделов языкознания: грамматики, лексикологии, семасиологии, стилистики, социолингвистики, психолингвистики и др.

Для ТП особое значение имеют данные коммуникативной лингвистики об особенностях процесса речевой коммуникации, специфике прямых и косвенных речевых актов, о соотношении выраженного и подразумеваемого смысла в высказывании и тексте, влиянии контекста и ситуации общения на понимание текста, других факторах, определяющих коммуникативное поведение человека.

Важным методом исследования в лингвистике перевода служит сопоставительный анализ перевода, т.е. анализ формы и содержания текста перевода в сопоставлении с формой и содержанием оригинала. Эти тексты представляют собой объективные факты, доступные наблюдению и анализу. В процессе перевода устанавливаются определенные отношения между двумя текстами на разных языках (текстом оригинала и текстом перевода). Сопоставляя такие тексты, можно раскрыть внутренний механизм перевода, выявить эквивалентные единицы, а также обнаружить изменения формы и содержания, происходящие при замене единицы оригинала эквивалентной ей единицей текста перевода. При этом возможно и сравнение двух или нескольких переводов одного и того же оригинала. Сопоставительный анализ переводов дает возможность выяснить, как преодолеваются типовые трудности перевода, связанные со спецификой каждого из языков, а также какие элементы оригинала остаются непереданными в переводе. В результа­те получается описание «переводческих фактов», дающее картину реального процесса.

Сопоставительное изучение переводов дает возможность по­лучать информацию о коррелятивности отдельных элементов оригинала и перевода, обусловленной как отношениями между языками, участвующими в переводе, так и внелингвистическими факторами, оказывающими влияние на ход переводческого процесса. Дополнительным методом получения такой информации может служить опрос информантов, в качестве которых используются лица, обладающие необходимым двуязычием и опытом переводческой деятельности. В процессе опроса ин­форманту предлагаются для перевода отрезки оригинала, содержащие лексические единицы или синтаксические структуры, представляющие определенные переводческие трудности.

