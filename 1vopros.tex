\section{Сущность и определение перевода. История перевода}

Слово «перевод» употребляется в смысле 'перевод с одного языка на другой', и в этом случае имеет два разных значения:\\
1) «Перевод как результат определенного процесса», то есть обозначение самого переведенного текста (напр., в предложениях: «Это --- очень хороший перевод романа Диккенса», «Недавно вышел в свет новый перевод поэмы Байрона «Паломничество Чайльд-Гарольда» на русский язык», «Он читал этого автора в переводе» и т. п.\\
2) «Перевод как сам процесс», то есть как действие от глагола переводить, в результате которого появляется текст перевода в первом значении.

%предмет науки о переводе
%Термин «процесс» применительно к переводу понимается нами в чисто лингвистическом смысле, то есть, как определенного вида языковое, точнее, межъязыковое преобразование или трансформация текста на одном языке в текст на другом языке. Опять-таки, термин «преобразование» нельзя понимать буквально --- сам исходный текст или текст оригинала не «преобразуется» в том смысле, что он не изменяется сам по себе. Этот текст, конечно, сам остается неизменным, но наряду с ним и на основе его создается другой текст на ином языке, который мы называем «переводом» в первом смысле этого слова (перевод как сам переведенный текст). Иными словами, термин «преобразование» (или «трансформация») здесь может быть употреблен лишь в том смысле, в каком этот термин применяется в синхронном описании языка вообще: речь идет об определенном отношении между двумя языковыми или речевыми единицами, из которых одна является исходной, а вторая создается на основе первой. В данном случае, имея исходный текст а на языке А, переводчик, применяя к нему определенные операции («переводческие трансформации», о которых речь пойдет ниже), создает текст на языке Б, который находится в определенных закономерных отношениях с текстом А. В своей совокупности эти языковые (межъязыковые) операции и составляют то, что мы называем «процессом перевода» в лингвистическом смысле. Таким образом, перевод можно считать определенным видом трансформации, а именно, межъязыковой трансформацией.

%Суммируя, можно сказать, что предметом лингвистической теории перевода является научное описание процесса перевода как межъязыковой трансформации, то есть преобразования текста на одном языке в эквивалентный ему текст на другом языке (о том, какое содержание вкладывается в термин «эквивалентный», речь пойдет ниже.) Иначе говоря, задачей лингвистической теории перевода является моделирование процесса перевода в указанном выше смысле.








Перевод --- текст, созданный на одном языке, выражается средствами дру­гого языка. При этом термин «текст» пони­мается предельно широко: имеется в виду любое устное высказыва­ние и любое письменное произведение от инструкции к холодильнику до романа.

Если полагать, что язык --- это своего рода код, т. е. произвольное обозначение предметов и явлений действительности с помощью ус­ловных знаков, то перевод можно назвать перекодированием, посколь­ку каждый из условных знаков заменяется при переводе знаком дру­гой знаковой системы.

Перевод есть перевыражение или перекодирование. Одна­ко это перекодирование не является объективным природным про­цессом, его осуществляет человек. Человек обладает индивидуаль­ностью и способностью к творчеству. Именно эти два фактора позволяют ему при перекодировании выбрать из нескольких или многих возможных вариантов перевода свой.

Рассмотренные параметры понятия «перевод» касались пре­имущественно описания его как процесса, но в ходе обсуждения его как процесса стало очевидным, что тем же термином мы обозначаем и результат этого процесса. Таким образом, в качестве рабочего оп­ределения понятия «перевод» можно принять следующее:

Перевод --- это деятельность, которая заключается в вариативном перевыражении, перекодировании текста, порожденного на од­ном языке, в текст на другом языке, осуществляемая переводчиком, который творчески выбирает вариант в зависимости от вариативных ресурсов языка, вида перевода, задач перевода, типа текста и под воздействием собственной индивидуальности; перевод --- это также и результат описанной выше деятельности.

Переводом называется процесс преобразования речевого произведения на одном языке в речевое произведение на другом языке при сохранении неизменного плана содержания, то есть значения.

1) Термин «план содержания» или «значение» следует понимать максимально широко, имея в виду все виды отношений, в которых находится знаковая (в данном случае, языковая) единица. 

2) О «сохранении неизменного плана содержания» можно говорить только в относительном, но не в абсолютном смысле. При межъязыковом преобразовании неизбежны потери, то есть имеет место неполная передача значений, выражаемых текстом подлинника. Стало быть, текст перевода никогда не может быть полным и абсолютным эквивалентом текста подлинника; задача переводчика заключается в том, чтобы сделать эту эквивалентность как можно более полной, то есть добиваться сведения потерь до минимума, но требовать «стопроцентного» совпадения значений, выражаемых в тексте подлинника и тексте перевода, было бы абсолютно нереальным. %Это значит также, что одной из задач теории перевода является установление того, что можно назвать порядком очередности передачи значений: учитывая, что существуют различные типы значений, необходимо установить, какие из них пользуются преимуществом при передаче в процессе перевода, а какими можно «жертвовать» с тем, чтобы семантические потери при переводе были минимальными.

Сама возможность сохранения плана содержания, то есть неизменности значения при переводе (хотя бы и относительной) предполагает, что в разных языках содержатся единицы, совпадающие по значению. Здесь правомерно задать вопрос: насколько справедливо это предположение? Если значение является, как мы предполагаем, неотъемлемой частью знака и, стало быть, единиц языка, то не значит ли это, что каждой знаковой системе, в том числе каждому языку, присущи свои специфические значения? И не вытекает ли из этого, что при преобразовании текста на одном языке в текст на другом языке, то есть, в процессе перевода неизбежно должны меняться не только языковые формы, но и выражаемые ими значения? На каком же основании мы тогда говорим, что значение в процессе перевода должно оставаться неизменным?

Вопрос этот весьма серьезен и заслуживает подробного рассмотрения. Расхождения в семантических системах разных языков --- несомненный факт, являющийся источником многочисленных трудностей, возникающих перед переводчиком в процессе осуществления перевода. Многие исследователи на этом основании считают возможным утверждать, что эквивалентность подлинника и перевода не базируется на тождестве выражаемых значений. Из многочисленных высказываний на эту тему процитируем только одно, принадлежащее английскому теоретику перевода Дж. Кэтфорду: «...Мнение о том, что текст на ИЯ и текст на ПЯ «имеют одно и то же значение» или что при переводе происходит «перенос значения», лишено оснований. Значение, с нашей точки зрения, есть свойство определенного языка. Текст на ИЯ имеет значение, свойственное ИЯ, в то время как текст на ПЯ имеет значение, свойственное ПЯ; например, русский текст имеет русское значение (точно также, как и русскую фонологию или графологию, грамматику и лексику), а эквивалентный ему английский текст имеет английское значение».

\subsection*{История перевода}

Рождение перевода, как особого вида деятельности человека, уходит корнями в глубокую древность, еще в дописьменные времена, когда между разноязычными племенами начались первые контакты. Как развивался перевод в те далекие времена нам остается лишь догадываться.

Появление переводчиков-профессионалов, возрастание значения переводческой деятельности, можно отнести ко времени возникновения ранних государств на Древнем Востоке и установления между ними различного рода отношений (торговых, политических и пр.).

Древность

Самое древнее изображение переводчика было обнаружено на египетском барельефе III тысячелетия до н. э., а самый первый известный по имени переводчик --- египтянин, верховный жрец в Тинисе (XIV в. до н.э.) --- Анхурмес.

В древнеегипетских текстах XXVIII--XX вв. до н.э. многократно говорится о «начальнике переводчиков», что означает существование в Древнем Египте особых профессиональных переводческих групп с собственной иерархией.

Первые письменные памятники перевода в Египте относятся к XV в. до н.э. Это дипломатическая переписка, переведенная с древнеегипетского языка на аккадский клинописью. Нет сомнений, что деятельность переводчиков Древнего Египта была достаточно интенсивной, однако, как показали исследования, в основном ее задачи были информационно-коммуникативными. Не установлено существование литературного перевода, а также нотариально заверенного перевода в древнеегипетской цивилизации.

Также высока была культура перевода в Древнем Шумере. Это древнее государство было завоевано Вавилоном, однако оказало на своих завоевателей значительное культурное влияние. Это способствовало установлению и развитию равно как и информационно-коммуникативной, так и литературной переводческой деятельности. Огромную роль сыграли также шумерские школы. Их выпускники переводили с шумерского языка на аккадский и наоборот, устно и письменно.

Культура Вавилона многое приняла из шумерской культуры. Множество шумерских текстов были переведены на аккадский язык. Переводчики же назывались «драгоманами» (targamannu(m)) и «сепиру». Существовала особая должностная иерархия переводчиков в Вавилоне: царские, храмовые, военные переводчики и т.д.

Имеются также данные о деятельности переводчиков в Хеттском государстве (XVIII--XIII вв. до н.э.). Сохранились переводы с хурритского языка на хеттский в письменной форме, а также шумеро-аккадо-хеттские словари --- свидетельства контактов с прочими древними культурами.

Античность

Уникальность культуры античности обусловлена огромнейшим потенциалом творчества, развитым Грецией и Римом за очень малый по меркам истории отрезок времени --- VI--I вв. до н.э. Отражение этого --- богатейшее наследие письменной литературы.

Литература Древней Греции, единственная из европейских литератур, которая развивалась самостоятельно. Известно, что греки с высокомерием относились к другим народам и, соответственно, презрительно к их «варварским» языкам. Таким образом, греческая литература в свой классический период V--IV вв. до н.э. вообще не знала художественного перевода. У греков, безусловно, существовали контакты с внешним миром, но обеспечивались они переводчиками-наемниками, «варварами», знающими греческий язык.

Значительные следы переводческой деятельности мы встречаем в поздний период Древней Греции и после распада великой державы после смерти Александра Македонского. Греческий язык получил широкое распространение и сохранил свое значение и после того, как оказался под властью Рима.

Рим на раннем этапе своего становления испытывал сильное влияние греческой культуры, и основной объем переводов осуществлялся с греческого языка. Знание греческого языка было свидетельством образованности и высокого общественного статуса, а контакты дипломатического характера с Грецией осуществлялись посредством перевода, который выполняли влиятельные граждане Рима.

Самые первые из известных заключений, касающихся вопросов перевода, сделаны именно в Древнем Риме. Тогда же начали появляться и первые оценки переводов.

Например, Марк Тулий Цицерон (106--43 гг. до н.э.) считал, что буквальный, дословный перевод является признаком языковой бедности и беспомощности переводчика. Он говорил, что необходимо передавать не форму, а смысл содержания.

Средневековье

В Средние века письменный перевод выполнялся монахами и считался богоугодным занятием. Перевод основывался на понятии иконической природы слова, т. е слово --- сакрально, а связь между словом и вещью --- нерасторжима. Поэтому переводы осуществлялись пословно, буквально. Возникающая в результате некая непонятность содержания усиливала мистическое религиозное чувство. Греческий и латынь, с которых переводили, понимались как божественные. Богослужения проходили на латыни. Однако устный перевод, напротив, понимался занятием от беса. (Умение говорить сразу на двух, тем более нескольких языках непременно наводило на мысль о связи с дьяволом). Тем не менее, незаменимость устных переводчиков при различных дипломатических контактах постепенно придавала переводчикам все большую значимость и признание. В XIII веке в Париже создается первая Высшая школа устных переводчиков.

Перевод библейской литературы

Согласно преданию, царь Египта Птолемей (правление 285-246 гг. до.н.э.) приказал перевести текст Библии с языка еврейского на язык греческий, для чего из каждого из двенадцати «колен Израилевых» было отобрано по 6 переводчиков. Переводчиков доставили на остров Фарос и на время работы исключили всякие контакты их друг с другом. По завершении работ оказалось, что все 72 рукописи были идентичны. Эта версия перевода получила название Септуагинта (лат. Septuaginta --- «семьдесят»), или переводом Семидесяти Толковников (т.е. переводчиков).

Первым переводом Библии на латынь считается перевод начала II в. н. э., который называют Вульгата, выполненный Иеронимом Стридонским (340?--420 гг.).

В IV в. н. э. Библия переводится на сирийский язык, на эфиопский и готский языки (Библия Вульфилы (317--381 гг.)).

Перевод в Европе XIV--XVI вв.

В эпоху Возрождения постепенно на перевод начинают смотреть иначе. Появляется больше переводов светской литературы. Переводчики XV--XVI вв. отмечают, что тексту перевода следует передавать не только смысл оригинала, но и соответствовать нормам переводящего языка.

В XVI в. в эпоху Реформации немецкий теолог и священник Мартин Лютер (1483--1546) осуществляет новый перевод Священного писания. Лютер также призывает помимо соблюдения точности передачи содержания Библии переводить так, чтобы любой человек мог понять язык, опираться на нормы языка.

История отечественного перевода

На Руси, как и в Европе, с X в. традиция перевода основывалась на понятии иконической природы слова. Самыми первыми были переводы со среднегреческого языка на старославянский --- язык православного богослужения. В XI--XII вв. по числу переводов Русь опережает все славянские государства. Однако, помимо религиозной литературы, переводилась также и светская.

В XVI в. на смену пословному переводу приходит концепция грамматического перевода, основанная на особенностях структуры переводящего языка, связанная с именем Максима Грека, монаха из Афона. Максим Грек переводил, большей частью книги религиозного характера, а также нерелигиозные тексты, исправлял уже существующие переводы, комментировал их.

Петр I осознавал всю важность развития переводческого дела для будущего становления России как державы, для развития ее культуры. Некоторые исторические данные также говорят о том, что сам царь был и переводчиком, и критиком. Петр имел собственное представление о переводе как о виде деятельности, собственное отношение к переводу, осознание того, для каких целей и как он должен выполняться. Он был резким противником буквального перевода и главным считал передачу смысла.

На первом этапе для перевода отбирались произведения, имевшие полезность и ценность для державы, способствовавшие развитию экономики и науки.

В 1724 году Петр I издал указ о создании Академии: «Учинить академию, в которой бы учились языкам, а также прочим наукам и знатным художествам и переводили бы книги».

Особое значение в развитии переводческой деятельности имело «Российское собрание» (1735 г.), созданное при Академии, которое явилось, по сути, первой организацией российских переводчиков-профессионалов.

При Екатерине II (1762--1796) очень много стало осуществляться переводов художественной литературы. Императрица, владела несколькими языками и занималась переводами.

В 1735--1783 гг. существовало Российское собрание, где переводчики собирались, обсуждали переводы, теоретические и практические вопросы переводческого дела.

В 19-м веке деятельность переводчика стала считаться высоким искусством, в первую очередь, благодаря таким выдающимся талантам, как А. Карамзин, В. Жуковский, М. Лермонтов.

После революции 1917 года по инициативе М.Горького было создано новое издательство «Всемирная литература». Тогда задумалось исправить и издать переводы всех крупных произведений западных и восточных писателей. В следующие несколько десятков лет издательство сумело осуществить задуманное, невзирая на большие трудности. Были опубликованы переводы книг многих знаменитых писателей и поэтов --- Бальзака, Франса, Стендаля, Гейне, Шиллера, Байрона, Диккенса, Б. Шоу, Марка Твена и многих, многих других. В работе были заняты выдающиеся литераторы и переводчики.








