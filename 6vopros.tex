\section{Единица перевода, контекст и нормы ПЯ как факторы, влияющие на процесс и результат перевода}

Единица перевода

Под единицей перевода мы имеем в виду такую единицу в исходном тексте, которой может быть найдено соответствие в тексте перевода, составные части которой по отдельности не имеют соответствий в тексте перевода. Иначе говоря, единица перевода --- минимальный фрагмент текста, требующий принятия переводческого решения. Единица перевода --- это минимальная языковая единица текста оригинала, воспринимаемая как единое целое с точки зрения семантики.

В современном языкознании принято различать следующие уровни языковой иерархии:
\\
---   уровень фонем (для письменной речи --- графем);
\\
---   уровень морфем;
\\
---   уровень слов;
\\
---   уровень словосочетаний;
\\
---   уровень предложений;
\\
---   уровень текста.

1. Перевод на уровне фонем/графем

Фонема, как известно, не является носителем самостоятельного значения, она играет в языке лишь смыслоразличительную роль. Тем не менее в переводе единицей перевода иногда оказывается именно фонема. Это происходит при использовании приемов транскрипции и транслитерации. Перевод на уровне фонем регулярно встречается: при передаче личных имен и географических названий, при транскрипционном способе заимствования слов других языков, обозначающих новые понятия, при передаче звукописи стиха. 

Транслитерация --- передача написания иноязычного слова с помощью средств ПЯ (Amsterdam --- Амстердам).

Транскрипция --- передача звучания иноязычного слова с помощью средств ИЯ (New-York --- Нью-Йорк).

2. Перевод на уровне морфем

Единицей перевода оказывается морфема, когда морфологическая структура семантически эквивалентных слов в разных языках совпадает. Например, это происходит при переводе композитных слов с учетом значения корневых морфем (moonlight --- лунный свет) и на уровне служебных морфем. Такое же поморфемное соответствие наблюдается и при переводе английского backbencher русским заднескамеечник.

3. Перевод на уровне слов

Слово выступает в качестве единицы перевода, когда каждому слову в переводе находится пословное соответствие (Не can swim - Он умеет плавать), или когда слову в ПЯ ему соответствует несколько слов (Она возвращается --- She comes back). Это происходит довольно часто при передаче простых, элементарных по структуре предложений.

4. Перевод на уровне словосочетания

Словосочетание считается единицей перевода, когда переводчик работает с семантическим единством как на уровне языка (фразеологизмы, парные словосочетания, формулы контакта, устойчивые наименования организаций, аббревиатуры), так и на уровне речи (реализация контекстуального значения слова). Напр., англ, to catch fire --- русск. загореться; англ, first night --- русск. премьера. 

5. Перевод на уровне предложений

Семантическим единством на уровне предложения обладают пословицы: (Rome wasn't built in a day. --- He сразу Москва строилась).

Единицей перевода является предложение и при переводе устойчивых клише и формул: надписей, сигнальных знаков, формул вежливости (Never drink unboiled water. --- He пейте сырой воды).

6. Перевод на уровне текста

Текст в качестве единицы перевода обычно рассматривают на примере поэзии. Не только строгие по построению стихотворно-композиционные формы, такие, как сонет, но и лирические стихи свободной архитектоники переводятся исходя из семантического единства всего произведения. Также некоторые рекламные тексты переводятся только целиком. Это не исключает того, что в таких текстах некоторые особенности оригинала передаются с помощью единиц перевода меньшего объема.

Контекст

Перевод не предполагает механической подстановки соответствия вместо переводимой единицы оригинала. Контексты определяют:
\\
--- выбор того или иного соответствия;
\\
--- отказ от известных соответствий;
\\
--- необходимость поиска иных способов перевода.

Лингвистический контекст --- языковое окружение той или иной единицы языка в тексте (совокупность слов, грамматических форм и конструкций).

Узкий контекст (микроконтекст) --- в пределах предложения; подразделяется на синтаксический и лексический.

Широкий контекст (макроконтекст) --- текстовый, без точных рамок.

Ситуативный контекст (экстралингвистический) -- обстановка, время и место, к которому относится высказывание; любые факты реальной действительности, знание которых помогает рецептору и переводчику правильно интерпретировать значения языковых единиц в высказывании.

Использование языковых соответствий всегда предполагает учет контекста.

Например: The striking unions have won concessions. Все полнозначные слова в этом предложении вне контекста имели бы по несколько значений (strike --- бить, бастовать; win --- убедить, победить, добиться). Сопоставив эти значения в контексте высказывания, поймем, что они  совместимы лишь будучи взяты в определенном значении.